\documentclass{beamer}
\usepackage[utf8]{inputenc}
\usepackage[brazil]{babel}
% \usepackage{mathabx}
\usepackage{mathpazo}
% \usepackage{eulervm}
% \usepackage{natbib}
\usepackage{multicol}
\usepackage{multirow}

%========================
\usepackage{colortbl}
\usepackage{tikz}
\usepackage{tikzit}
\usepackage{graphicx}% http://ctan.org/pkg/graphicx
\usepackage{booktabs}% http://ctan.org/pkg/booktabs
\usepackage{pgfplots}
\pgfplotsset{compat=1.9}
\usetikzlibrary{plotmarks}
\usepackage{tikzsymbols}
\tikzstyle{basic}=[fill=black, draw=black, shape=circle]
\tikzstyle{empty}=[draw=black, shape=circle]

\setbeamertemplate{theorems}[numbered]
\setbeamertemplate{bibliography item}{}
% \newtheorem{definition}{Definição}
\newtheorem{observation}{Observação}
\newtheorem{properties}{Propriedades}
\newtheorem{corolary}{Corolário}
\newtheorem{conjecture}{Conjectura}
% \newtheorem{theorem}{Teorema}
\newtheorem{proposition}{Proposição}
% \newtheorem{lemma}{Lema}
% \newtheorem{fact}{Fato}
% \setbeamertemplate{lemmas}[numbered]
\newtheorem{lema}{Lema}
\newtheorem{teo}{Teorema}
\newtheorem{fato}{Fato}

\providecommand{\tightlist}{%
  \setlength{\itemsep}{0pt}\setlength{\parskip}{0pt}}

% Estilo para grafos
\pgfplotsset{compat=1.9}
\usetikzlibrary{plotmarks,arrows,shapes,backgrounds,calc,positioning,matrix}
\tikzstyle{basic}=[fill=black, draw=black, shape=circle,minimum size=1em,inner sep=0pt]
\tikzstyle{ccinza}=[fill=black!50, draw=black, shape=circle,minimum size=1em,inner sep=0pt]
\tikzstyle{contaminado}=[fill=black!95, draw=black, shape=circle,minimum size=1em,inner sep=0pt]
\tikzstyle{empty}=[draw=black, shape=circle,minimum size=1em,inner sep=0pt]
\tikzstyle{cvermelho}=[fill=red, draw=black, shape=circle,minimum size=1em,inner sep=0pt]
\tikzstyle{cverde}=[fill=green, draw=black, shape=circle,minimum size=1em,inner sep=0pt]
\tikzstyle{camarelo}=[fill=yellow, draw=black, shape=circle,minimum size=1em,inner sep=0pt]
\tikzstyle{cazul}=[fill=blue, draw=black, shape=circle,minimum size=1em,inner sep=0pt]

\tikzset{
  invisible/.style={opacity=0},
  visible on/.style={alt={#1{}{invisible}}},
  alt/.code args={<#1>#2#3}{%
    \alt<#1>{\pgfkeysalso{#2}}{\pgfkeysalso{#3}} % \pgfkeysalso doesn't change the path
  },
}



\usetheme{Dresden}
\usefonttheme{serif}
\usecolortheme{rose}


\title{Número de $P_3$-Carathéodory em Grafos Circulantes}
\author{Braully Rocha da Silva, Erika Morais Martins Coelho, {\bf Hebert Coelho da Silva}}
\institute{Universidade Federal de Goiás \par Instituto de Informática}

% \usepackage[backend=biber,style=authoryear]{biblatex}
% \bibliography{artigo-p3-hull-circulant}
% \addbibresource{artigo-p3-hull-circulant.bib}

% \titlegraphic{
% \includegraphics[width=3.5cm]{img/logo-ppgc.png}
% \vspace*{0.5cm}
% }

\begin{document}

% \maketitle
\begin{frame}
  \titlepage
  \centering
  \includegraphics[width=3.5cm]{img/logo-ppgc.png}
\end{frame}

%[ ] Logo INF e UFG
\begin{frame}{Roteiro}
  \begin{itemize}
    \item Motivação
    \item Preliminares
    \item Resultados
    \item Conclusão e trabalhos futuros
  \end{itemize}
\end{frame}

\section{Motivação}
\begin{frame}{Modelo de propagação}
      O processo de influência pode ser modelado pelo contexto:

      \begin{itemize}
            \tightlist
            \item
                  Alguns indivíduos estão inicialmente influenciados;
            \item
                  Os demais indivíduos são influenciados à medida que seus vizinhos
                  também ficam influenciados.
            \item
                  Novos indivíduos influenciados podem propagar a influência a outros
                  indivíduos.
      \end{itemize}
      % \begin{figure}
      % \centering
      % \includegraphics[scale=0.2]{wa.png}
      % \end{figure}
\end{frame}

\begin{frame}{Problemas relacionados}
      \begin{itemize}
            \tightlist
            \item
                  Propagação de informações em redes sociais
            \item
                  Disseminação de notícias falsas
            \item
                  Marketing viral
            \item
                  Infecção e doenças contagiosas

      \end{itemize}
\end{frame}

\section{Preliminares}
\begin{frame}{Notações}
      \begin{columns}[T]
            \begin{column}{.7\textwidth}
                  \emph{Grafo simples}: \(G = (V(G)), E(G))\)

                  \emph{Notações}:
                  \begin{itemize}
                        \tightlist
                        \item \(N(v)=\{u \in V(G)|vu \in E(G)\}\)
                        \item \(d(v)=|N(v)|\)
                  \end{itemize}

                  \vspace{0.5cm}
                  \uncover<2->{Exemplo: $v=5$}
                  \begin{itemize}
                        \item<3-> $N(v)=\{1,3\}$
                        \item<3-> $d(v)=2$
                  \end{itemize}
            \end{column}
            \begin{column}{.3\textwidth}
                  % \begin{tikzpicture}
	\begin{pgfonlayer}{nodelayer}
		\node [style=empty] (0) at (-1, 1) {1};
		\node [style=empty] (1) at (1, 1) {2};
		\node [style=empty] (2) at (-1, -1) {4};
		\node [style=empty] (3) at (1, -1) {3};
		\node [style=empty] (4) at (0, 0) {5};
	\end{pgfonlayer}
	\begin{pgfonlayer}{edgelayer}
		\draw (0) to (2);
		\draw (2) to (3);
		\draw (3) to (1);
		\draw (1) to (0);
		\draw (0) to (4);
		\draw (4) to (3);
	\end{pgfonlayer}
\end{tikzpicture}

                  \begin{tikzpicture}
                        \begin{pgfonlayer}{nodelayer}
                              \node [style=empty] (0) at (-1, 1) {1};
                              \node [style=empty] (1) at (1, 1) {2};
                              \node [style=empty] (2) at (-1, -1) {4};
                              \node [style=empty] (3) at (1, -1) {3};
                              \node [style=empty] (4) at (0, 0) {5};
                              %Cores
                              \node [style=cvermelho,visible on=<2->] () at (0, 0) {5};
                              \node [style=cazul,visible on=<3->] () at (-1, 1) {1};
                              \node [style=cazul,visible on=<3->] () at (1, -1) {3};

                        \end{pgfonlayer}
                        \begin{pgfonlayer}{edgelayer}
                              \draw (0) to (2);
                              \draw (2) to (3);
                              \draw (3) to (1);
                              \draw (1) to (0);
                              \draw (0) to (4);
                              \draw (4) to (3);
                              \draw[visible on=<3->,red] (0) to (4);
                              \draw[visible on=<3->,red] (4) to (3);
                        \end{pgfonlayer}
                  \end{tikzpicture}

            \end{column}
      \end{columns}

      %[] colcoar uma figura geral, e 
\end{frame}


\begin{frame}{Convexidade em Grafos}
      A \textit{convexidade} em um grafo $G$ é dada por uma coleção ${\cal C}$ de subconjuntos de $V(G)$ tal que:
      \begin{itemize}
            \item$\emptyset,V(G) \in {\cal C}$
            \item ${\cal C}$ é um conjunto fechado em relação a operação de intersecção
            \item Cada elemento de ${\cal C}$ é um \textit{conjunto convexo}
      \end{itemize}
\end{frame}

\begin{frame}{Convexidade $P_3$}
      Convexidades definidas por um conjunto ${\cal P}$ de caminhos em grafos.
      
      \vspace{0.5cm}
      \textit{Convexidade $P_3$} é quando ${\cal P}$ é o conjunto de todos os caminhos com três vértices.

\end{frame}

\begin{frame}{Convexidade $P_3$}

      %[ ] Observação: Colocar uma figura com um conjunto não convexo, e um overlay para um conjunto convexo.
      \begin{columns}[T]
            \begin{column}{0.5\textwidth}
                  % \begin{tikzpicture}
	\begin{pgfonlayer}{nodelayer}
		\node [style=empty] (0) at (-3, 0) {a};
		\node [style=empty] (1) at (-2, 1) {b};
		\node [style=empty] (2) at (-2, -1) {i};
		\node [style=empty] (3) at (0, 1) {d};
		\node [style=empty] (4) at (2, 1) {e};
		\node [style=empty] (5) at (0, -1) {g};
		\node [style=empty] (6) at (2, -1) {f};
		\node [style=empty] (7) at (-1, 2) {c};
		\node [style=empty] (8) at (-1, -2) {h};
	\end{pgfonlayer}
	\begin{pgfonlayer}{edgelayer}
		\draw (0) to (1);
		\draw (1) to (2);
		\draw (0) to (2);
		\draw (1) to (7);
		\draw (7) to (3);
		\draw (1) to (3);
		\draw (3) to (4);
		\draw (4) to (6);
		\draw (6) to (5);
		\draw (5) to (2);
		\draw (2) to (8);
		\draw (8) to (5);
		\draw (3) to (5);
	\end{pgfonlayer}
\end{tikzpicture}

                  \begin{tikzpicture}
                        \begin{pgfonlayer}{nodelayer}
                              \node [style=empty] (0) at (-3, 0) {a};
                              \node [style=empty] (1) at (-2, 1) {b};
                              \node [style=empty] (2) at (-2, -1) {i};
                              \node [style=empty] (3) at (0, 1) {d};
                              \node [style=empty] (4) at (2, 1) {e};
                              \node [style=empty] (5) at (0, -1) {g};
                              \node [style=empty] (6) at (2, -1) {f};
                              \node [style=empty] (7) at (-1, 2) {c};
                              \node [style=empty] (8) at (-1, -2) {h};
                              %convexo
                              % b, c, d
                              \node [style=cazul,visible on=<{2,2}>] () at (-2, 1) {b};
                              \node [style=cazul,visible on=<{2,2}>] () at (-1, 2) {c};
                              \node [style=cazul,visible on=<{2,2}>] () at (0, 1) {d};
                              % b, d, g
                              \node [style=cvermelho,visible on=<3->] () at (-2, 1) {b};
                              \node [style=cvermelho,visible on=<3->] () at (0, 1) {d};
                              \node [style=cvermelho,visible on=<3->] () at (0, -1) {g};
                        \end{pgfonlayer}
                        \begin{pgfonlayer}{edgelayer}
                              \draw (0) to (1);
                              \draw (1) to (2);
                              \draw (0) to (2);
                              \draw (1) to (7);
                              \draw (7) to (3);
                              \draw (1) to (3);
                              \draw (3) to (4);
                              \draw (4) to (6);
                              \draw (6) to (5);
                              \draw (5) to (2);
                              \draw (2) to (8);
                              \draw (8) to (5);
                              \draw (3) to (5);
                        \end{pgfonlayer}
                  \end{tikzpicture}
            \end{column}
            \begin{column}{0.5\textwidth}
                  Seja $G=(V,E)$ um grafo, um conjunto $S$ é convexo na convexidade $P_3$ quando todo vértice em $V\setminus S$  tem no máximo um vizinho em $S$.

                  \begin{itemize}
                        \item<{2,3}> $S_1=\{b,c,d\}$ é convexo
                        \item<{3,3}>  $S_2=\{b,d,g\}$ não é convexo
                  \end{itemize}
            \end{column}
      \end{columns}
\end{frame}

\begin{frame}{Convexidade $P_3$}
      \begin{columns}[T]
            \begin{column}{0.5\textwidth}
                  % \begin{tikzpicture}
	\begin{pgfonlayer}{nodelayer}
		\node [style=empty] (0) at (-3, 0) {a};
		\node [style=empty] (1) at (-2, 1) {b};
		\node [style=empty] (2) at (-2, -1) {i};
		\node [style=empty] (3) at (0, 1) {d};
		\node [style=empty] (4) at (2, 1) {e};
		\node [style=empty] (5) at (0, -1) {g};
		\node [style=empty] (6) at (2, -1) {f};
		\node [style=empty] (7) at (-1, 2) {c};
		\node [style=empty] (8) at (-1, -2) {h};
	\end{pgfonlayer}
	\begin{pgfonlayer}{edgelayer}
		\draw (0) to (1);
		\draw (1) to (2);
		\draw (0) to (2);
		\draw (1) to (7);
		\draw (7) to (3);
		\draw (1) to (3);
		\draw (3) to (4);
		\draw (4) to (6);
		\draw (6) to (5);
		\draw (5) to (2);
		\draw (2) to (8);
		\draw (8) to (5);
		\draw (3) to (5);
	\end{pgfonlayer}
\end{tikzpicture}

                  \begin{tikzpicture}
                        \begin{pgfonlayer}{nodelayer}
                              \node [style=empty] (0) at (-3, 0) {a};
                              \node [style=empty] (1) at (-2, 1) {b};
                              \node [style=empty] (2) at (-2, -1) {i};
                              \node [style=empty] (3) at (0, 1) {d};
                              \node [style=empty] (4) at (2, 1) {e};
                              \node [style=empty] (5) at (0, -1) {g};
                              \node [style=empty] (6) at (2, -1) {f};
                              \node [style=empty] (7) at (-1, 2) {c};
                              \node [style=empty] (8) at (-1, -2) {h};
                              %não convexo
                              % b, d, g
                              \node [style=cvermelho] () at (-2, -1) {i};
                              \node [style=cvermelho] () at (0, -1) {g};
                              \node [style=cazul,visible on=<2->] () at (-1, -2) {h};

                        \end{pgfonlayer}
                        \begin{pgfonlayer}{edgelayer}
                              \draw (0) to (1);
                              \draw (1) to (2);
                              \draw (0) to (2);
                              \draw (1) to (7);
                              \draw (7) to (3);
                              \draw (1) to (3);
                              \draw (3) to (4);
                              \draw (4) to (6);
                              \draw (6) to (5);
                              \draw (5) to (2);
                              \draw (2) to (8);
                              \draw (8) to (5);
                              \draw (3) to (5);
                        \end{pgfonlayer}
                  \end{tikzpicture}
            \end{column}
            \begin{column}{0.5\textwidth}
                  \emph{Intervalo fechado}: para $u,v \in V(G)$ é o conjunto $I[u, v]$ de todos os vértices pertencentes a todo caminho $P_3$ entre $u$ e $v$.
                  %[ ] Fazer o overlay para o u e v e o intervalo fechado de u e v.
                  % [ ] Mesma figura fazer o intervalo fechado de um conjunto S

                  \begin{itemize}
                        \item<{2,3}> $I[i,g]=\{i,g,h\}$
                        \item<{3,3}> Se $S \subseteq V(G)$, então $I[S]$ é a união de todos $I[u, v]$ para $u, v \in S$.
                  \end{itemize}
            \end{column}
      \end{columns}
\end{frame}


\begin{frame}{Envoltória e percolação}
      \begin{columns}[T]
            \begin{column}{0.5\textwidth}
                  % \begin{tikzpicture}
	\begin{pgfonlayer}{nodelayer}
		\node [style=empty] (0) at (-3, 0) {a};
		\node [style=empty] (1) at (-2, 1) {b};
		\node [style=empty] (2) at (-2, -1) {i};
		\node [style=empty] (3) at (0, 1) {d};
		\node [style=empty] (4) at (2, 1) {e};
		\node [style=empty] (5) at (0, -1) {g};
		\node [style=empty] (6) at (2, -1) {f};
		\node [style=empty] (7) at (-1, 2) {c};
		\node [style=empty] (8) at (-1, -2) {h};
	\end{pgfonlayer}
	\begin{pgfonlayer}{edgelayer}
		\draw (0) to (1);
		\draw (1) to (2);
		\draw (0) to (2);
		\draw (1) to (7);
		\draw (7) to (3);
		\draw (1) to (3);
		\draw (3) to (4);
		\draw (4) to (6);
		\draw (6) to (5);
		\draw (5) to (2);
		\draw (2) to (8);
		\draw (8) to (5);
		\draw (3) to (5);
	\end{pgfonlayer}
\end{tikzpicture}

                  \begin{tikzpicture}
                        \begin{pgfonlayer}{nodelayer}
                              \node [style=empty] (0) at (-3, 0) {a};
                              \node [style=empty] (1) at (-2, 1) {b};
                              \node [style=empty] (2) at (-2, -1) {i};
                              \node [style=empty] (3) at (0, 1) {d};
                              \node [style=empty] (4) at (2, 1) {e};
                              \node [style=empty] (5) at (0, -1) {g};
                              \node [style=empty] (6) at (2, -1) {f};
                              \node [style=empty] (7) at (-1, 2) {c};
                              \node [style=empty] (8) at (-1, -2) {h};
                              %convexo
                              % b, c, d
                              \node [style=cvermelho,visible on=<2->] () at (-2, 1) {b};
                              \node [style=cvermelho,visible on=<2->] () at (0, 1) {d};
                              \node [style=cvermelho,visible on=<2->] () at (0, -1) {g};
                              %i1
                              \node [style=cazul,visible on=<3->] () at (-1, 2) {c};
                              \node [style=cazul,visible on=<3->] () at (-2, -1) {i};
                              %i2
                              \node [style=cazul,visible on=<4->] () at (-3, 0) {a};
                              \node [style=cazul,visible on=<4->] () at (-1, -2) {h};



                        \end{pgfonlayer}
                        \begin{pgfonlayer}{edgelayer}
                              \draw (0) to (1);
                              \draw (1) to (2);
                              \draw (0) to (2);
                              \draw (1) to (7);
                              \draw (7) to (3);
                              \draw (1) to (3);
                              \draw (3) to (4);
                              \draw (4) to (6);
                              \draw (6) to (5);
                              \draw (5) to (2);
                              \draw (2) to (8);
                              \draw (8) to (5);
                              \draw (3) to (5);
                        \end{pgfonlayer}
                  \end{tikzpicture}
            \end{column}
            \begin{column}{0.5\textwidth}
                  \emph{Envoltória convexa}: $H(S) = I^{k}[S]$ tal que $I^k[S] = I^{k-1}[S]$ para $k \geq 2$.

                  
                  \emph{Percolação}: O processo de computar a envoltória convexa de um conjunto não convexo.

                  \begin{itemize}
                        \item<2-> $S=\{b,d,g\}$
                        \item<3-> $I^1[S]=S \cup \{c,i\}$.
                        \item<4-> $I^2[S]=I^1[S] \cup \{a,h\}$.
                        \item<5-> $I^3[S]=I^2[S]$.
                        \item<5-> $H(S)=I^3[S]=I^2[S]$.

                  \end{itemize}
            \end{column}
      \end{columns}
\end{frame}


\begin{frame}{Conjunto de Carathéodory}
      \begin{columns}[T]
            \begin{column}{0.4\textwidth}
                  % \begin{tikzpicture}
	\begin{pgfonlayer}{nodelayer}
		\node [style=empty] (0) at (-1.75, 1.75) {4};
		\node [style=empty] (1) at (-1.75, 0.75) {5};
		\node [style=empty] (2) at (-0.75, 1.5) {3};
		\node [style=empty] (3) at (0.5, 1.25) {2};
		\node [style=empty] (4) at (1.75, 1.5) {1};
		\node [style=empty] (5) at (0.25, 0.25) {6};
		\node [style=empty] (6) at (1.5, -0.5) {8};
		\node [style=empty] (7) at (-0.5, -0.5) {7};
		\node [style=empty] (8) at (-1.5, -1) {9};
		\node [style=empty] (9) at (-0.75, -1.5) {10};
	\end{pgfonlayer}
	\begin{pgfonlayer}{edgelayer}
		\draw (1) to (2);
		\draw (0) to (2);
		\draw (2) to (3);
		\draw (3) to (4);
		\draw (3) to (5);
		\draw (5) to (6);
		\draw (5) to (7);
		\draw (7) to (8);
		\draw (7) to (9);
	\end{pgfonlayer}
\end{tikzpicture}

                  \begin{tikzpicture}
                        \begin{pgfonlayer}{nodelayer}
                              \node [style=empty] (0) at (-1.75, 1.75) {4};
                              \node [style=empty] (1) at (-1.75, 0.75) {5};
                              \node [style=empty] (2) at (-0.75, 1.5) {3};
                              \node [style=empty] (3) at (0.5, 1.25) {2};
                              \node [style=empty] (4) at (1.75, 1.5) {1};
                              \node [style=empty] (5) at (0.25, 0.25) {6};
                              \node [style=empty] (6) at (1.5, -0.5) {8};
                              \node [style=empty] (7) at (-0.5, -0.5) {7};
                              \node [style=empty] (8) at (-1.5, -1) {9};
                              \node [style=empty] (9) at (-0.75, -1.5) {0};
                              %S=1,4,5,8,9,0
                              \node [style=ccinza] () at (1.75, 1.5) {1};
                              \node [style=ccinza] () at (-1.75, 1.75) {4};
                              \node [style=ccinza] () at (-1.75, 0.75) {5};
                              \node [style=ccinza] () at (1.5, -0.5) {8};
                              \node [style=ccinza] () at (-1.5, -1) {9};
                              \node [style=ccinza] () at (-0.75, -1.5) {0};
                              % 3,2,6,7
                              \node [style=cazul,visible on=<{2,2}>] () at (-0.75, 1.5) {3};
                              \node [style=cazul,visible on=<{2,2}>] () at (0.5, 1.25) {2};
                              \node [style=cazul,visible on=<{2,2}>] () at (0.25, 0.25) {6};
                              \node [style=cazul,visible on=<{2,2}>] () at (-0.5, -0.5) {7};
                              %v=7
                              \node [style=cazul,visible on=<{3,3}>] () at (-0.5, -0.5) {7};
                              %v=3
                              \node [style=cazul,visible on=<{4,4}>] () at (-0.75, 1.5) {3};
                              %v=6
                              \node [style=cazul,visible on=<{5,5}>] () at (0.25, 0.25) {6};
                              %v=2
                              \node [style=cazul,visible on=<{6,6}>] () at (0.5, 1.25) {2};
                              

                        \end{pgfonlayer}
                        \begin{pgfonlayer}{edgelayer}
                              \draw (1) to (2);
                              \draw (0) to (2);
                              \draw (2) to (3);
                              \draw (3) to (4);
                              \draw (3) to (5);
                              \draw (5) to (6);
                              \draw (5) to (7);
                              \draw (7) to (8);
                              \draw (7) to (9);
                        \end{pgfonlayer}
                  \end{tikzpicture}

                  Qual o minimo do $|S|$ tal que $v\in H(S)$, para todo $v\in H(S) \setminus S$.
            \end{column}
            \begin{column}{0.6\textwidth}
                  \begin{itemize}
                        \item<2-> $H(S)=V(G)$
                        \item<3-> Se $v=7$, então $v\in H(\{9,0\})$
                        \item<4-> Se $v=3$, então $v\in H(\{4,5\})$
                        \item<5-> Se $v=6$, então $v\in H(\{8,9,0\})$
                        \item<6-> Se $v=2$, então $v\in H(\{1,4,5\})$
                        \item<7-> Então, $c_s(G)=3$
                        \item<8-> Para calcular o $c(G)$, é necessário calcular o $c_s(G)$ para todo $S \subseteq V(G)$ e pegar o maior valor.
                  \end{itemize}
            \end{column}
      \end{columns}
\end{frame}

\begin{frame}
      \frametitle{Nº de Carathéodory}


      \begin{itemize}
      \item{O \textit{número de Carathéodory} $c(G)$ é o menor inteiro $c$,
                        para o qual todo $u \in H(S)$, existe um conjunto $F \subseteq  S$,
                        com $|F| \le c$ e $u \in H(F)$}
                        
            \item{Se $\partial H(S)=H(S) \setminus \bigcup _{u \in S} H(S \setminus \{u\})$,
                        é não vazio, então $S$ é um \textit{conjunto de Carathéodory}}
            
            \item{Esta definição permite uma  forma alternativa,
                        o número de Carathéodory é a maior cardinalidade de um conjunto de Carathéodory}
      \end{itemize}
\end{frame}

% \begin{frame}{Caratheódory}
%       \begin{columns}[T]
%             \begin{column}{0.5\textwidth}
%                   \begin{tikzpicture}
	\begin{pgfonlayer}{nodelayer}
		\node [style=empty] (0) at (-3, -1) {e};
		\node [style=empty] (1) at (-1, -1) {f};
		\node [style=empty] (2) at (1, -1) {g};
		\node [style=empty] (3) at (3, -1) {h};
		\node [style=empty] (4) at (3, 1) {d};
		\node [style=empty] (5) at (1, 1) {c};
		\node [style=empty] (6) at (-1, 1) {b};
		\node [style=empty] (7) at (-3, 1) {a};
	\end{pgfonlayer}
	\begin{pgfonlayer}{edgelayer}
		\draw (4) to (3);
		\draw (3) to (2);
		\draw (5) to (2);
		\draw (2) to (1);
		\draw (1) to (0);
		\draw (1) to (6);
		\draw (0) to (7);
		\draw (7) to (6);
	\end{pgfonlayer}
\end{tikzpicture}

%             \end{column}
%             \begin{column}{0.5\textwidth}
%                   \begin{itemize}
%                         \item H
%                   \end{itemize}
%             \end{column}
%       \end{columns}
% \end{frame}


\begin{frame}{Grafos circulantes}
      \textbf{Definição:} Os grafos circulantes $C_n(L)$ são grafos circulares com $n$ vértices e lista de inteiros $L$, tal que $C_n(L)$ tem vértices $V(C_n(L))=\{1,2,...,n\}$ e arestas $st\in E(C_n(L))| (s+t) \bmod n \in L$

      \vspace{0.5cm}

\end{frame}

\begin{frame}{Exemplo de grafos circulantes}
\textbf{Construção simplificada:} Para cada $j\in L$ o $i$-ésimo vértice é adjacente ao ($i+j$)-ésimo e ao ($i-j$)-ésimo vértice.
\vspace{0.3cm}

 \centering
      \resizebox{\textwidth}{!}{%
            \begin{tikzpicture}
                  \begin{pgfonlayer}{nodelayer}
                        \node [style=empty, label={$v_1$}] (0) at (3.5, 2) {};
                        \node [style=empty, label={below:$v_4$}] (1) at (3.5, -2) {};
                        \node [style=empty, label={below:$v_3$}] (2) at (5.5, -1) {};
                        \node [style=empty, label={$v_2$}] (3) at (5.5, 1) {};
                        \node [style=empty, label={$v_6$}] (4) at (1.5, 1) {};
                        \node [style=empty, label={below:$v_5$}] (5) at (1.5, -1) {};
                        % 
                        \node [style=empty, label={$v_1$}] (14) at (-3.5, 2) {};
                        \node [style=empty, label={$v_2$}] (15) at (-1.5, 1) {};
                        \node [style=empty, label={below:$v_3$}] (16) at (-1.5, -1) {};
                        \node [style=empty, label={below:$v_4$}] (17) at (-3.5, -2) {};
                        \node [style=empty, label={below:$v_5$}] (18) at (-5.5, -1) {};
                        \node [style=empty, label={$v_6$}] (19) at (-5.5, 1) {};

                        % Labels G1
                        \node [style=none,visible on=<{1,1}>] (20) at (-3.5, -4) {$C_6(2)$};
                        \node [style=none,visible on=<{2,2}>] (20) at (-3.5, -4) {$C_6(\textbf{2})=\{v_1v_3,v_2v_4,v_4v_6,v_6v_2\}$};
                        \node [style=none,visible on=<{3,4}>] (20) at (-3.5, -4) {$C_6(2)$};

                        % Labels G2
                        \node [style=none,visible on=<{1,1}>] (21) at (3.5, -4) {$C_6(1,2)$};
                        \node [style=none,visible on=<{2,2}>] (21) at (3.5, -4) {$C_6(\textbf{1},2)=\{v_1v_2,v_3v_4,v_5v_6,v_6v_1\}$};
                        \node [style=none,visible on=<{3,3}>] (21) at (3.5, -4) {$C_6(1,\textbf{2})=\{v_1v_2,v_3v_4,v_5v_6,v_6v_1,v_1v_3,v_2v_4,v_4v_6,v_6v_2\}$};
                        \node [style=none,visible on=<{4,4}>] (21) at (3.5, -4) {$C_6(1,2)$};
                  \end{pgfonlayer}
                  \begin{pgfonlayer}{edgelayer}
                        %Ciclo C6(1)
                        \draw[visible on=<2->]  (0) to (3){};
                        \draw[visible on=<2->] (3) to (2);
                        \draw[visible on=<2->] (2) to (1);
                        \draw[visible on=<2->] (1) to (5);
                        \draw[visible on=<2->] (5) to (4);
                        \draw[visible on=<2->] (4) to (0);

                        % Finalização C6(1,2)
                        \draw[visible on=<3->] (0) to (2);
                        \draw[visible on=<3->] (3) to (1);
                        \draw[visible on=<3->] (2) to (5);
                        \draw[visible on=<3->] (1) to (4);
                        \draw[visible on=<3->] (5) to (0);
                        \draw[visible on=<3->] (4) to (3);

                        % Pentagrama C6(2)
                        \draw[visible on=<2->] (14) to (16);
                        \draw[visible on=<2->] (18) to (16);
                        \draw[visible on=<2->] (18) to (14);
                        \draw[visible on=<2->] (17) to (15);
                        \draw[visible on=<2->] (19) to (15);
                        \draw[visible on=<2->] (19) to (17);
                  \end{pgfonlayer}
            \end{tikzpicture}
      }
\end{frame}



\begin{frame}{Trabalhos Relacionados}
\begin{itemize}
    \item NP-difícil: Determinar se um grafo $G$ tem um conjunto Carathéodory de tamanho $k$ [Barbosa et al. (2012)].
    \item Algoritmo de tempo polinomial para árvores, co-grafos [Barbosa et al. (2012)] e cordais [Coelho et al. (2014)]
    \item Limite superior para número envoltória de grafos circulantes [Shaheen et al.(2022)].
\end{itemize}
\end{frame}




% \begin{frame}{Propriedades Caratheódory}
%       Barbosa et al. (2012) apresentam algumas propriedades de conjuntos de Carathéodory em grafos.
%       \begin{proposition}
%             Seja $G$ um grafo e $S$ um conjunto de Carathéodory de $G$.
%             \begin{enumerate}
%                   \def\labelenumi{\Alph{enumi}.}
%                   \item{
%                               Se $G$ tem ordem de pelo menos 2 e é completo ou um caminho ou um ciclo, então $c(G)=2$.
%                         }
%                   \item{
%                               Se $S$ tem ordem pelo menos dois, então todo vértice de $u$ em $S$ está em um caminho $uvw$ de ordem 3 tal que $v\in V(G) \backslash H_G(S\backslash \{u\})$ e $w \in H_G(S\backslash \{u\})$
%                         }
%                   \item{
%                               Nenhum subconjunto próprio $S'$ de $S$ satisfaz  $H_G(S')=V(G)$.
%                         }
%                   \item{
%                               O fecho convexo $H_G(S)$ de $S$ induz um subgrafo conexo de $G$.
%                         }
%             \end{enumerate}
%       \end{proposition}
% \end{frame}


\begin{frame}{Propriedades envoltória circulante}
      Shaheen et al. (2022) apresentam algumas propriedades de grafos circulantes que podem concluir a seguinte proposição.
      \begin{proposition}
            \label{prop-circulante-p3}
            Seja $G$ um grafo circulante $C_n(1,r)$ e $S$ um conjunto de vértices tal que $S=\{v_i,v_{i+1},v_{i+2},...,v_{i+(r-1)}\}$, então $S$ é um conjunto envoltório.
      \end{proposition}
\end{frame}

\begin{frame}{Exemplo Proposição \ref{prop-circulante-p3}}
      \centering
      \begin{tikzpicture}
            \begin{pgfonlayer}{nodelayer}
                  \node [style=basic, label={$v_1$}] (8) at (0.75, 2) {};
                  \node [style=basic, label={$v_2$}] (9) at (2, 0.75) {};
                  \node [style=basic, label={below:$v_3$}] (13) at (2, -0.75) {};
                  \node [style=empty, label={below:$v_4$}] (10) at (0.75, -2) {};
                  \node [style=empty, label={below:$v_5$}] (12) at (-0.75, -2) {};
                  \node [style=empty, label={below:$v_6$}] (11) at (-2, -0.75) {};
                  \node [style=empty, label={$v_7$}] (6) at (-2, 0.75) {};
                  \node [style=empty, label={$v_8$}] (7) at (-0.75, 2) {};

                  % I1
                  \node [style=camarelo, visible on=<2->] () at (-0.75, 2) {};
                  \node [style=cazul, visible on=<2->] () at (0.75, -2) {};
                  % I2
                  \node [style=cazul, visible on=<3->] () at (-0.75, -2) {};
                  \node [style=camarelo, visible on=<3->] () at (-2, 0.75) {};
                  % I3
                  \node [style=cverde, visible on=<2->] () at (-2, -0.75) {};
                  % Label
                  \node [style=none, visible on=<{1,1}>] () at (0, -3) {$C_8(1,3)$ e $S=\{v_1,v_2,v_3\}$};
                  \node [style=none, visible on=<{2,2}>] () at (0, -3) {$I^1[S]=I[S]\cup \{v_4,v_6,v_8\} $};
                  \node [style=none, visible on=<{3,3}>] () at (0, -3) {$I^2[S]=I^1[S] \cup \{v_5,v_7\}$};
                  \node [style=none, visible on=<{4,4}>] () at (0, -3) {$I^3[S]=I^2[S]=H(S)$};
                  % \node [style=none, visible on=<{5,5}>] () at (0, -3) {$I^3[S]=I^2[S]=H(S)$};
            \end{pgfonlayer}
            \begin{pgfonlayer}{edgelayer}
                  \draw (8) to (9);
                  \draw (9) to (13);
                  \draw (13) to (10);
                  \draw (10) to (12);
                  \draw (12) to (11);
                  \draw (11) to (6);
                  \draw (6) to (7);
                  \draw (7) to (8);
                  \draw (8) to (10);
                  \draw (9) to (12);
                  \draw (13) to (11);
                  \draw (10) to (6);
                  \draw (12) to (7);
                  \draw (11) to (8);
                  \draw (6) to (9);
                  \draw (7) to (13);
            \end{pgfonlayer}
      \end{tikzpicture}

\end{frame}
\section{Resultados}
\begin{frame}{Para um grafo circulante $C_n(1,2)$}
    \begin{fato}
        \label{fact-carat}
        Seja $G$ um grafo e $S$ um conjunto de Carathéodory de $G$, então $S$ possui pelo menos dois vértices $u$ e $v$ tal que $w\in N(u)\cap N(v)$, onde $w \in V(G)$.
    \end{fato}

    \begin{lema}
        \label{lemma-circ-1-2}
        Seja $G = C_n(1,2)$, então $c(G)=2$.
    \end{lema}

    \begin{proposition}
        Seja $C_n^k$ um grafo potência de ciclo, então $c(C_n^k) = 2$.
    \end{proposition}
\end{frame}

% \begin{frame}{Para $C_n(1,r)$ tal que $3 | (r+1)$}

% \end{frame}


\begin{frame}{}
    \begin{lema}
        \label{lemma-circ-1-r}
        Considere o grafo $G = C_n(1,r)$ tal que $3 | (r+1)$ e $S \subseteq V(G)$. Se $S = \{v_i \;|\; 3 \nmid i$ para $1\le i \le r\}$ então $v_{2r} \in H(S)$.
    \end{lema}
    
    \begin{columns}[T]
        \begin{column}{0.7\textwidth}
            \resizebox{\textwidth}{!}{
                \begin{tikzpicture}
	\begin{pgfonlayer}{nodelayer}
		\node [style=basic, label={$v_1$}] (0) at (-4, 1.25) {};
		\node [style=basic, label={$v_2$}] (1) at (-3, 1.25) {};
		\node [style=empty, label={$v_3$}] (2) at (-2, 1.25) {};
		\node [style=basic, label={$v_4$}] (3) at (-1, 1.25) {};
		\node [style=basic, label={$v_5$}] (4) at (0, 1.25) {};
		\node [style=empty, label={$v_6$}] (16) at (1, 1.25) {};
		\node [style=basic, label={$v_7$}] (18) at (2, 1.25) {};
		\node [style=basic, label={$v_{r-1}$}] (10) at (4, 1.25) {};
		\node [style=basic, label={$v_{r}$}] (13) at (5, 1.25) {};
		%Propagação
		\node [style=empty, label={below:$v_{r+1}$}] (5) at (-4, 0) {};
		\node [style=empty, label={below:$v_{r+2}$}] (6) at (-3, 0) {};
		\node [style=empty, label={below:$v_{r+3}$}] (7) at (-2, 0) {};
		
		\node [style=empty, label={below:$v_{r+4}$}] (8) at (-1, 0) {};
		\node [style=empty, label={below:$v_{r+5}$}] (9) at (0, 0) {};
		\node [style=empty, label={below:$v_{r+6}$}] (17) at (1, 0) {};
		\node [style=empty, label={below:$v_{r+7}$}] (19) at (2, 0) {};
		%...
		\node [style=empty, label={below:$v_{2r-2}$}] (15) at (3, 0) {};
		\node [style=empty, label={below:$v_{2r-1}$}] (11) at (4, 0) {};
		\node [style=empty, label={below:$v_{2r}$}] (12) at (5, 0) {};


		\node [style=empty, label={$v_{r-2}$}] (14) at (3, 1.25) {};

		%I1
		%, label={$v_3$}, label={$v_6$}, label={$v_{r-2}$}, label={below:$v_{r+1}$}
		\node [style=cazul, visible on=<2->] () at (-2, 1.25) {};
		\node [style=cazul, visible on=<2->] () at (1, 1.25) {};
		\node [style=cazul, visible on=<2->] () at (3, 1.25) {};
		\node [style=cazul, visible on=<2->] () at (-4, 0) {};
		%I2 
		%, label={below:$v_{r+2}$}
		\node [style=cazul, visible on=<3->] () at (-3, 0) {};
		%I3
		% , label={below:$v_{r+3}$}
		\node [style=cazul, visible on=<4->] () at (-2, 0) {};
		%I
		% V2r
		\node [style=cvermelho, visible on=<5->] () at (5, 0) {};
		% Labels, label={below:$v_{r+4}$}, label={below:$v_{r+5}$}, label={below:$v_{r+6}$}
		%, label={below:$v_{r+7}$}, label={below:$v_{2r-2}$}, label={below:$v_{2r-1}$}
		\node [style=cazul, visible on=<5->] () at (-1, 0) {};
		\node [style=cazul, visible on=<5->] () at (0, 0) {};
		\node [style=cazul, visible on=<5->] () at (1, 0) {};
		\node [style=cazul, visible on=<5->] () at (2, 0) {};
		%...
		\node [style=cazul, visible on=<5->] () at (3, 0) {};
		\node [style=cazul, visible on=<5->] () at (4, 0) {};


	\end{pgfonlayer}
	\begin{pgfonlayer}{edgelayer}
		\draw (0) to (5);
		\draw (1) to (6);
		\draw (2) to (7);
		\draw (3) to (8);
		\draw (4) to (9);
		\draw (0) to (1);
		\draw (1) to (2);
		\draw (2) to (3);
		\draw (3) to (4);
		\draw (5) to (6);
		\draw (6) to (7);
		\draw (7) to (8);
		\draw (8) to (9);
		\draw (10) to (13);
		\draw (13) to (12);
		\draw (12) to (11);
		\draw (10) to (11);
		\draw (13) to (5);
		\draw (14) to (10);
		\draw (15) to (11);
		\draw (14) to (15);
		\draw (16) to (17);
		\draw (18) to (19);
		\draw (4) to (16);
		\draw (16) to (18);
		\draw (9) to (17);
		\draw (17) to (19);
		\draw[dashed] (19) to (15);
		\draw[dashed] (18) to (14);
	\end{pgfonlayer}
\end{tikzpicture}

            }



        \end{column}
        \begin{column}{0.3\textwidth}
            \begin{itemize}
                \item<2-> $\{ v_i \; | \; 3 \mid i \} \in I^1[S]$.
                \item<3-> $v_{r+2} \in I^2[S]$.
                \item<4-> $v_{r+3} \in I^3[S]$.
                \item<5-> $\cdots$
                \item<5-> $v_{2r} \in H(S)$.
            \end{itemize}
        \end{column}
    \end{columns}
\end{frame}

% \begin{frame}{Para $C_n(1,r)$ tal que $3 | (r+1)$}

% \end{frame}


\begin{frame}{}
    \begin{lema}
        \label{lemma-carat-1-rx3}
        Seja $G = C_n(1,r)$ tal que $3 | (r+1)$ e $S = \{v_i \in V(G) \;|\; 3 \nmid i$ para $1\le i \le r\}$. Se $n \ge 4r - 2$ então $ v_{2r} \in  \partial H(S)$.
    \end{lema}
    
    \begin{columns}
        \begin{column}{0.7\textwidth}
            \resizebox{\textwidth}{!}{
                \begin{tikzpicture}
	\begin{pgfonlayer}{nodelayer}
		\node [style=basic, label={$v_1$}] (0) at (-4, 1.25) {};
		\node [style=basic, label={$v_2$}] (1) at (-3, 1.25) {};
		\node [style=empty, label={$v_3$}] (2) at (-2, 1.25) {};
		\node [style=basic, label={$v_4$}] (3) at (-1, 1.25) {};
		\node [style=basic, label={$v_5$}] (4) at (0, 1.25) {};
		\node [style=empty, label={$v_6$}] (16) at (1, 1.25) {};
		\node [style=basic, label={$v_7$}] (18) at (2, 1.25) {};
		\node [style=basic, label={$v_{r-1}$}] (10) at (4, 1.25) {};
		\node [style=basic, label={$v_{r}$}] (13) at (5, 1.25) {};
		%Propagação
		\node [style=empty, label={below:$v_{r+1}$}] (5) at (-4, 0) {};
		\node [style=empty, label={below:$v_{r+2}$}] (6) at (-3, 0) {};
		\node [style=empty, label={below:$v_{r+3}$}] (7) at (-2, 0) {};
		
		\node [style=empty, label={below:$v_{r+4}$}] (8) at (-1, 0) {};
		\node [style=empty, label={below:$v_{r+5}$}] (9) at (0, 0) {};
		\node [style=empty, label={below:$v_{r+6}$}] (17) at (1, 0) {};
		\node [style=empty, label={below:$v_{r+7}$}] (19) at (2, 0) {};
		%...
		\node [style=empty, label={below:$v_{2r-2}$}] (15) at (3, 0) {};
		\node [style=empty, label={below:$v_{2r-1}$}] (11) at (4, 0) {};
		\node [style=empty, label={below:$v_{2r}$}] (12) at (5, 0) {};


		\node [style=empty, label={$v_{r-2}$}] (14) at (3, 1.25) {};

		%I1
		%, label={$v_3$}, label={$v_6$}, label={$v_{r-2}$}, label={below:$v_{r+1}$}
		\node [style=cazul, visible on=<2->] () at (-2, 1.25) {};
		\node [style=cazul, visible on=<2->] () at (1, 1.25) {};
		\node [style=cazul, visible on=<2->] () at (3, 1.25) {};
		\node [style=cazul, visible on=<2->] () at (-4, 0) {};
		%I2 
		%, label={below:$v_{r+2}$}
		\node [style=cazul, visible on=<3->] () at (-3, 0) {};
		%I3
		% , label={below:$v_{r+3}$}
		\node [style=cazul, visible on=<4->] () at (-2, 0) {};
		%I
		% V2r
		\node [style=cvermelho, visible on=<5->] () at (5, 0) {};
		% Labels, label={below:$v_{r+4}$}, label={below:$v_{r+5}$}, label={below:$v_{r+6}$}
		%, label={below:$v_{r+7}$}, label={below:$v_{2r-2}$}, label={below:$v_{2r-1}$}
		\node [style=cazul, visible on=<5->] () at (-1, 0) {};
		\node [style=cazul, visible on=<5->] () at (0, 0) {};
		\node [style=cazul, visible on=<5->] () at (1, 0) {};
		\node [style=cazul, visible on=<5->] () at (2, 0) {};
		%...
		\node [style=cazul, visible on=<5->] () at (3, 0) {};
		\node [style=cazul, visible on=<5->] () at (4, 0) {};


	\end{pgfonlayer}
	\begin{pgfonlayer}{edgelayer}
		\draw (0) to (5);
		\draw (1) to (6);
		\draw (2) to (7);
		\draw (3) to (8);
		\draw (4) to (9);
		\draw (0) to (1);
		\draw (1) to (2);
		\draw (2) to (3);
		\draw (3) to (4);
		\draw (5) to (6);
		\draw (6) to (7);
		\draw (7) to (8);
		\draw (8) to (9);
		\draw (10) to (13);
		\draw (13) to (12);
		\draw (12) to (11);
		\draw (10) to (11);
		\draw (13) to (5);
		\draw (14) to (10);
		\draw (15) to (11);
		\draw (14) to (15);
		\draw (16) to (17);
		\draw (18) to (19);
		\draw (4) to (16);
		\draw (16) to (18);
		\draw (9) to (17);
		\draw (17) to (19);
		\draw[dashed] (19) to (15);
		\draw[dashed] (18) to (14);
	\end{pgfonlayer}
\end{tikzpicture}

            }
            \resizebox{\textwidth}{!}{
                \input{img/escada-lemma-circ-1-r-reverso2-apres}
            }
            % Para:
            % \begin{itemize}
            %     \tightlist
            %     \item $C_n(1,r)$ com $3 | (r+1)$ e $n \ge 4\times r - 2$
            %     \item $S = \{v_i \in V(G) \;|\; 3 \nmid i$ para $1\le i \le r\}$.
            % \end{itemize}
        \end{column}
        \begin{column}{0.3\textwidth}
            % Quando $n \ge 4\times r - 2$ e $S = \{v_i \in V(G) \;|\; 3 \nmid i$ para $1\le i \le r\}$ temos:
            \footnotesize
            \begin{itemize}
                \tightlist
                \item<2-> $I^1[S]=S\cup \{ v_i \; | \; 3 \mid i \} \cup \{v_{r+1}, v_n \}$
                \item<3-> $I^2[S]=I^1[S]\cup\{v_{r+2}, v_{n-1}\}$
                \item<4-> $I^3[S]=I^2[S]\cup\{v_{r+3}, v_{n-2}\}$
                \item<5-> $I^{r}[S]=I^{r-1}[S]\cup\{v_{2r}, v_{n-(r-1)}\}$
                % \item<6-> $v_{2r} \in H(S)$ e $v_{2r} \not\in H(S-j)|\forall j\in S$
                % \item<7-> Logo $v_{2r} \in \partial H(S)$
            \end{itemize}
        \end{column}
    \end{columns}
\end{frame}


\begin{frame}{Para $G=C_n(1,r)$ tal que $3 | (r+1)$}
    \begin{teo}
        \label{theorem-carat-1-rx3}
        Seja $G = C_n(1,r)$ tal que $3 | (r+1)$ e $n \ge 4\times r - 2$ então $c(G) \ge \frac{2(r+1)}{3}$.%então $G$ possui um conjunto de Carathéodory de cardinalidade $ \frac{2(r+1)}{3} $. % \rceil \frac{2r}{3} \lceil$
    \end{teo}
    % \begin{proof}
    %     O conjunto $S = \{v_i \;|\; 3 \nmid i$ para $1\le i \le r\}$ pelo Lema~\ref{lemma-circ-1-r} temos que $v_{2r} \in H(S)$, e pelo Lema~\ref{lemma-carat-1-rx3} temos que $v_{2r} \in \partial H(S)$, portanto $S$ é um conjunto de Carathéodory de $G$. O conjunto $S$ tem cardinalidade $\frac{2(r+1)}{3}$ consequentemente o número de Carathéodory de $G$ é maior ou igual $\frac{2(r+1)}{3} $.  %$r - \lfloor \frac{r}{3} \rfloor = \rceil \frac{2r}{3} \lceil$.
    % \end{proof}
\end{frame}


\begin{frame}{Exemplo}
    \begin{figure}
        \centering
        \resizebox{\textwidth}{!}{
            \centering
            \begin{tikzpicture}%[scale=0.6]
	\begin{pgfonlayer}{nodelayer}
		\node [style=empty, label={$1$}] (17) at (-7.64591, 3.27063) {};
		\node [style=basic, label={above right:$2$}] (1) at (-6.56093, 2.85267) {};
		\node [style=basic, label={right:$3$}] (0) at (-5.68223, 2.09788) {};
		\node [style=basic, label={right:$4$}] (2) at (-5.11995, 1.08449) {};
		\node [style=basic, label={right:$5$}] (16) at (-4.93823, -0.05503) {};
		\node [style=empty, label={right:$6$}] (15) at (-5.14918, -1.19562) {};
		\node [style=empty, label={below right:$7$}] (12) at (-5.73964, -2.18778) {};
		\node [style=empty, label={below right:$8$}] (11) at (-6.63631, -2.9267) {};
		\node [style=empty, label={below:$9$}] (8) at (-7.73279, -3.30467) {};
		\node [style=empty, label={below:$10$}] (6) at (-8.8849, -3.29608) {};
		\node [style=empty, label={below left:$11$}] (4) at (-9.97193, -2.8879) {};
		\node [style=empty, label={below left:$12$}] (3) at (-10.852, -2.12699) {};
		\node [style=empty, label={left:$13$}] (5) at (-11.4274, -1.11805) {};
		\node [style=empty, label={left:$14$}] (7) at (-11.6071, 0.02817) {};
		\node [style=empty, label={left:$15$}] (9) at (-11.3829, 1.16353) {};
		\node [style=empty, label={left:$16$}] (10) at (-10.7934, 2.15672) {};
		\node [style=empty, label={above left:$17$}] (13) at (-9.90118, 2.89467) {};
		\node [style=empty, label={$18$}] (14) at (-8.80853, 3.28571) {};
		\node [style=none] (18) at (-8.25, 0) {$H(S-\{1\})$};
		\node [style=basic, label={$1$}] (19) at (0.52296, 3.27063) {};
		\node [style=empty, label={above right:$2$}] (20) at (1.60794, 2.85267) {};
		\node [style=empty, label={right:$3$}] (21) at (2.48664, 2.09788) {};
		\node [style=basic, label={right:$4$}] (22) at (3.04892, 1.08449) {};
		\node [style=basic, label={right:$5$}] (23) at (3.23064, -0.05503) {};
		\node [style=basic, label={right:$6$}] (24) at (3.01969, -1.19562) {};
		\node [style=empty, label={below right:$7$}] (25) at (2.42923, -2.18778) {};
		\node [style=empty, label={below right:$8$}] (26) at (1.53256, -2.9267) {};
		\node [style=empty, label={below:$9$}] (27) at (0.43608, -3.30467) {};
		\node [style=empty, label={below:$10$}] (28) at (-0.71603, -3.29608) {};
		\node [style=empty, label={below left:$11$}] (29) at (-1.80303, -2.8879) {};
		\node [style=empty, label={below left:$12$}] (30) at (-2.68313, -2.12699) {};
		\node [style=empty, label={left:$13$}] (31) at (-3.25853, -1.11805) {};
		\node [style=empty, label={left:$14$}] (32) at (-3.43823, 0.02817) {};
		\node [style=empty, label={left:$15$}] (33) at (-3.21403, 1.16353) {};
		\node [style=empty, label={left:$16$}] (34) at (-2.62453, 2.15672) {};
		\node [style=basic, label={above left:$17$}] (35) at (-1.73233, 2.89467) {};
		\node [style=basic, label={$18$}] (36) at (-0.63966, 3.28571) {};
		\node [style=none] (37) at (-0.08113, 0) {$H(S-\{2\})$};
		\node [style=basic, label={$1$}] (38) at (8.69183, 3.27063) {};
		\node [style=basic, label={above right:$2$}] (39) at (9.77681, 2.85267) {};
		\node [style=empty, label={right:$3$}] (40) at (10.6555, 2.09788) {};
		\node [style=empty, label={right:$4$}] (41) at (11.2178, 1.08449) {};
		\node [style=basic, label={right:$5$}] (42) at (11.3995, -0.05503) {};
		\node [style=basic, label={right:$6$}] (43) at (11.1886, -1.19562) {};
		\node [style=basic, label={below right:$7$}] (44) at (10.5981, -2.18778) {};
		\node [style=empty, label={below right:$8$}] (45) at (9.70143, -2.9267) {};
		\node [style=empty, label={below:$9$}] (46) at (8.60495, -3.30467) {};
		\node [style=empty, label={below:$10$}] (47) at (7.45284, -3.29608) {};
		\node [style=empty, label={below left:$11$}] (48) at (6.36584, -2.8879) {};
		\node [style=empty, label={below left:$12$}] (49) at (5.48574, -2.12699) {};
		\node [style=empty, label={left:$13$}] (50) at (4.91034, -1.11805) {};
		\node [style=empty, label={left:$14$}] (51) at (4.73064, 0.02817) {};
		\node [style=empty, label={left:$15$}] (52) at (4.95484, 1.16353) {};
		\node [style=empty, label={left:$16$}] (53) at (5.54434, 2.15672) {};
		\node [style=empty, label={above left:$17$}] (54) at (6.43654, 2.89467) {};
		\node [style=basic, label={$18$}] (55) at (7.52921, 3.28571) {};
		\node [style=none] (56) at (8.08774, 0) {$H(S-\{4\})$};
	\end{pgfonlayer}
	\begin{pgfonlayer}{edgelayer}
		\draw (17) to (1);
		\draw (1) to (0);
		\draw (0) to (2);
		\draw (2) to (16);
		\draw (16) to (15);
		\draw (15) to (12);
		\draw (12) to (11);
		\draw (11) to (8);
		\draw (8) to (6);
		\draw (6) to (4);
		\draw (4) to (3);
		\draw (3) to (5);
		\draw (5) to (7);
		\draw (7) to (9);
		\draw (9) to (10);
		\draw (10) to (13);
		\draw (13) to (14);
		\draw (14) to (17);
		\draw (17) to (15);
		\draw (1) to (12);
		\draw (0) to (11);
		\draw (2) to (8);
		\draw (16) to (6);
		\draw (15) to (4);
		\draw (12) to (3);
		\draw (11) to (5);
		\draw (8) to (7);
		\draw (6) to (9);
		\draw (4) to (10);
		\draw (3) to (13);
		\draw (5) to (14);
		\draw (7) to (17);
		\draw (9) to (1);
		\draw (10) to (0);
		\draw (13) to (2);
		\draw (19) to (20);
		\draw (20) to (21);
		\draw (21) to (22);
		\draw (22) to (23);
		\draw (23) to (24);
		\draw (24) to (25);
		\draw (25) to (26);
		\draw (26) to (27);
		\draw (27) to (28);
		\draw (28) to (29);
		\draw (29) to (30);
		\draw (30) to (31);
		\draw (31) to (32);
		\draw (32) to (33);
		\draw (33) to (34);
		\draw (34) to (35);
		\draw (35) to (36);
		\draw (36) to (19);
		\draw (19) to (24);
		\draw (20) to (25);
		\draw (21) to (26);
		\draw (22) to (27);
		\draw (23) to (28);
		\draw (24) to (29);
		\draw (25) to (30);
		\draw (26) to (31);
		\draw (27) to (32);
		\draw (28) to (33);
		\draw (29) to (34);
		\draw (30) to (35);
		\draw (31) to (36);
		\draw (32) to (19);
		\draw (33) to (20);
		\draw (34) to (21);
		\draw (35) to (22);
		\draw (38) to (39);
		\draw (39) to (40);
		\draw (40) to (41);
		\draw (41) to (42);
		\draw (42) to (43);
		\draw (43) to (44);
		\draw (44) to (45);
		\draw (45) to (46);
		\draw (46) to (47);
		\draw (47) to (48);
		\draw (48) to (49);
		\draw (49) to (50);
		\draw (50) to (51);
		\draw (51) to (52);
		\draw (52) to (53);
		\draw (53) to (54);
		\draw (54) to (55);
		\draw (55) to (38);
		\draw (38) to (43);
		\draw (39) to (44);
		\draw (40) to (45);
		\draw (41) to (46);
		\draw (42) to (47);
		\draw (43) to (48);
		\draw (44) to (49);
		\draw (45) to (50);
		\draw (46) to (51);
		\draw (47) to (52);
		\draw (48) to (53);
		\draw (49) to (54);
		\draw (50) to (55);
		\draw (51) to (38);
		\draw (52) to (39);
		\draw (53) to (40);
		\draw (54) to (41);
		\draw (14) to (16);
		\draw (36) to (23);
		\draw (55) to (42);
	\end{pgfonlayer}
\end{tikzpicture}

        }
        \caption{Exemplo de um conjunto de Carathéodory para $C_{18}(1,5)$}
        \label{fig-c18-pt1}
    \end{figure}
\end{frame}


\begin{frame}{Exemplo}
    \begin{figure}
        \centering
        \resizebox{\textwidth}{!}{
            \centering
            \input{img/c18-pt2-final}
        }
        \caption{Exemplo de um conjunto de Carathéodory para $C_{18}(1,5)$}
        \label{fig-c18-pt2}
    \end{figure}
\end{frame}


\begin{frame}{Para $C_n(1,r)$ tal que $3\nmid(r+1)$}
    De forma análoga vamos concluir um limite inferior para o número de Carathéodory para $C_n(1,r)$, onde $r+1$ não é múltiplo de 3.
\end{frame}

% \begin{frame}{Para $C_n(1,r)$ tal que $3\nmid(r+1)$}
%     \begin{figure}
%     % \resizebox{\textwidth}{!}{
%         \centering
%         \input{img/exemplo-quando-pndv3}
%     % }
%     \caption{Exemplos de $S$ para quando $3\nmid (r+1)$ 
%     %conforme Lema\ref{lemma-carat-1-r-geral}
%     }
%     \label{fig-exemplo-lema-geral}
% \end{figure}
% \end{frame}

\begin{frame}{Para $C_n(1,r)$ tal que $3\nmid(r+1)$}
\begin{lema}
        \label{lemma-carat-1-r-geral}
        Seja $G=C_n(1,r)$ tal que $3\nmid(r+1)$ e $S\subseteq V(G)$. Se $S=\{v_r\}\cup \{ v_i | 3 \nmid i $ tal que $1\le i  < r-1 \}$ então $v_{2r} \in H(S)$.
    \end{lema}
    \begin{lema}
        \label{lemma-carat-1-rx3-geral}
        Seja $G = C_n(1,r)$ tal que $3 \nmid (r+1)$ e $S=\{ v_i | 3 \nmid i $ tal que $1\le i  < r-1 \}\cup \{v_r\}$  . Se $n \ge 4r - 2$ então $ v_{2r} \in  \partial H(S)$.
    \end{lema}

    \begin{teo}
        \label{theorem-carat-1-r-geral}
        Seja $G = C_n(1,r)$ tal que $3 \nmid (r+1)$ e $n \ge 4\times r - 2$  então $c(G) \ge \lfloor \frac{2r+1}{3} \rfloor$. %$G$ possui um conjunto de Carathéodory de cardinalidade $\lfloor \frac{2r+1}{3} \rfloor$. %$\frac{2r+1}{3}$.
    \end{teo}
\end{frame}
% \begin{frame}{Considerações finais e trabalhos futuros}
    \begin{itemize}
        \tightlist
        \item Determinar o limite superior
        \item Explorar o parâmetro número de Carathéodory para outras classes de grafo da família Cayley:
              \begin{itemize}
                  \tightlist
                  \item Hamming
                  \item Hipercubo
                  \item Kneser
              \end{itemize}
    \end{itemize}
\end{frame}




% \begin{frame}[allowframebreaks]{Referências}
% % \bibliographystyle{plain}
% % \bibliography{artigo-p3-hull-circulant}
% \printbibliography
% \end{frame}
\section{Conclusão e trabalhos futuros}
\begin{frame}{Conclusão}
  % $NP-$difícil determinar se um grafo possui um conjunto de Carathéodory de tamanho $k$ [Barbosa et al., 2012], determinamos um limite inferior para o número de Carathéodory em grafos circulantes na convexidade $P_3$:
  \begin{itemize}
    \item Para grafos circulantes $C_n(1,2)$ o número de Carathéodory é 2.
    \item Para grafos circulantes $C_n(1,r)$ com $r>2$ e $n\ge4r-2$:
          \begin{itemize}
            \item $c(C_n(1,r)) \ge \lfloor \frac{2r+1}{3} \rfloor$ se $3 \nmid (r+1)$
            \item $c(C_n(1,r)) \ge \frac{2(r+1)}{3} $ caso contrário.
          \end{itemize}
  \end{itemize}
\end{frame}

\begin{frame}{Trabalhos Futuros}
  \begin{itemize}
    \item Determinar limite superior para o número de Carathéodory em grafos circulantes na convexidade $P_3$.
    \item Estender os resultados para outros tipos de grafos de Cayley: Hipercubos, Hamming e Kneser.
  \end{itemize}

\end{frame}

\begin{frame}{Referências}
          \begin{itemize}
            \item{Barbosa et al. (2012): \textbf{On
                              the Carathéodory Number for the Convexity of Paths of Order Three}. SIAM Journal on Discrete
                        Mathematics, 26(3):929–939. ISSN 0895-4801.}
            \item{Coelho et al. (2014)}: \textbf{The Carathéodory number of the convexity of chordal graphs}. Discrete Applied Mathematics, vol. 172, 2014 p.104--108.

            \item{Shaheen et al.(2022): \textbf{Irreversible k -Threshold Conversion Number of
                              Circulant Graphs}. Journal of Applied Mathematics, 2022(1):1250951. ISSN 16870042.}
      \end{itemize}
\end{frame}


\begin{frame}{Fim}

  Dúvidas e sugestões?

  \begin{center}
    \Huge Obrigado!
    % [ ] Logo CNPQ, UFG e INF
    \begin{columns}[T]
      \begin{column}{0.5\textwidth}
        \vspace*{1cm}
        \resizebox{\textwidth}{!}{
          \includegraphics{img/logo-cnpq.png}
        }
      \end{column}
      \begin{column}{0.5\textwidth}
        \resizebox{\textwidth}{!}{
          \includegraphics{img/logo-ppgc.png}
        }
      \end{column}
    \end{columns}
  \end{center}


\end{frame}
\end{document}
