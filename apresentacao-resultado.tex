\section{Resultados}
\begin{frame}{Para um grafo circulante $C_n(1,2)$}
    \begin{fato}
        \label{fact-carat}
        Seja $G$ um grafo e $S$ um conjunto de Carathéodory de $G$, então $S$ possui pelo menos dois vértices $u$ e $v$ tal que $w\in N(u)\cap N(v)$, onde $w \in V(G)$.
    \end{fato}

    \begin{lema}
        \label{lemma-circ-1-2}
        Seja $G = C_n(1,2)$, então $c(G)=2$.
    \end{lema}

    \begin{proposition}
        Seja $C_n^k$ um grafo potência de ciclo, então $c(C_n^k) = 2$.
    \end{proposition}
\end{frame}

% \begin{frame}{Para $C_n(1,r)$ tal que $3 | (r+1)$}

% \end{frame}


\begin{frame}{}
    \begin{lema}
        \label{lemma-circ-1-r}
        Considere o grafo $G = C_n(1,r)$ tal que $3 | (r+1)$ e $S \subseteq V(G)$. Se $S = \{v_i \;|\; 3 \nmid i$ para $1\le i \le r\}$ então $v_{2r} \in H(S)$.
    \end{lema}
    
    \begin{columns}[T]
        \begin{column}{0.7\textwidth}
            \resizebox{\textwidth}{!}{
                \begin{tikzpicture}
	\begin{pgfonlayer}{nodelayer}
		\node [style=basic, label={$v_1$}] (0) at (-4, 1.25) {};
		\node [style=basic, label={$v_2$}] (1) at (-3, 1.25) {};
		\node [style=empty, label={$v_3$}] (2) at (-2, 1.25) {};
		\node [style=basic, label={$v_4$}] (3) at (-1, 1.25) {};
		\node [style=basic, label={$v_5$}] (4) at (0, 1.25) {};
		\node [style=empty, label={$v_6$}] (16) at (1, 1.25) {};
		\node [style=basic, label={$v_7$}] (18) at (2, 1.25) {};
		\node [style=basic, label={$v_{r-1}$}] (10) at (4, 1.25) {};
		\node [style=basic, label={$v_{r}$}] (13) at (5, 1.25) {};
		%Propagação
		\node [style=empty, label={below:$v_{r+1}$}] (5) at (-4, 0) {};
		\node [style=empty, label={below:$v_{r+2}$}] (6) at (-3, 0) {};
		\node [style=empty, label={below:$v_{r+3}$}] (7) at (-2, 0) {};
		
		\node [style=empty, label={below:$v_{r+4}$}] (8) at (-1, 0) {};
		\node [style=empty, label={below:$v_{r+5}$}] (9) at (0, 0) {};
		\node [style=empty, label={below:$v_{r+6}$}] (17) at (1, 0) {};
		\node [style=empty, label={below:$v_{r+7}$}] (19) at (2, 0) {};
		%...
		\node [style=empty, label={below:$v_{2r-2}$}] (15) at (3, 0) {};
		\node [style=empty, label={below:$v_{2r-1}$}] (11) at (4, 0) {};
		\node [style=empty, label={below:$v_{2r}$}] (12) at (5, 0) {};


		\node [style=empty, label={$v_{r-2}$}] (14) at (3, 1.25) {};

		%I1
		%, label={$v_3$}, label={$v_6$}, label={$v_{r-2}$}, label={below:$v_{r+1}$}
		\node [style=cazul, visible on=<2->] () at (-2, 1.25) {};
		\node [style=cazul, visible on=<2->] () at (1, 1.25) {};
		\node [style=cazul, visible on=<2->] () at (3, 1.25) {};
		\node [style=cazul, visible on=<2->] () at (-4, 0) {};
		%I2 
		%, label={below:$v_{r+2}$}
		\node [style=cazul, visible on=<3->] () at (-3, 0) {};
		%I3
		% , label={below:$v_{r+3}$}
		\node [style=cazul, visible on=<4->] () at (-2, 0) {};
		%I
		% V2r
		\node [style=cvermelho, visible on=<5->] () at (5, 0) {};
		% Labels, label={below:$v_{r+4}$}, label={below:$v_{r+5}$}, label={below:$v_{r+6}$}
		%, label={below:$v_{r+7}$}, label={below:$v_{2r-2}$}, label={below:$v_{2r-1}$}
		\node [style=cazul, visible on=<5->] () at (-1, 0) {};
		\node [style=cazul, visible on=<5->] () at (0, 0) {};
		\node [style=cazul, visible on=<5->] () at (1, 0) {};
		\node [style=cazul, visible on=<5->] () at (2, 0) {};
		%...
		\node [style=cazul, visible on=<5->] () at (3, 0) {};
		\node [style=cazul, visible on=<5->] () at (4, 0) {};


	\end{pgfonlayer}
	\begin{pgfonlayer}{edgelayer}
		\draw (0) to (5);
		\draw (1) to (6);
		\draw (2) to (7);
		\draw (3) to (8);
		\draw (4) to (9);
		\draw (0) to (1);
		\draw (1) to (2);
		\draw (2) to (3);
		\draw (3) to (4);
		\draw (5) to (6);
		\draw (6) to (7);
		\draw (7) to (8);
		\draw (8) to (9);
		\draw (10) to (13);
		\draw (13) to (12);
		\draw (12) to (11);
		\draw (10) to (11);
		\draw (13) to (5);
		\draw (14) to (10);
		\draw (15) to (11);
		\draw (14) to (15);
		\draw (16) to (17);
		\draw (18) to (19);
		\draw (4) to (16);
		\draw (16) to (18);
		\draw (9) to (17);
		\draw (17) to (19);
		\draw[dashed] (19) to (15);
		\draw[dashed] (18) to (14);
	\end{pgfonlayer}
\end{tikzpicture}

            }



        \end{column}
        \begin{column}{0.3\textwidth}
            \begin{itemize}
                \item<2-> $\{ v_i \; | \; 3 \mid i \} \in I^1[S]$.
                \item<3-> $v_{r+2} \in I^2[S]$.
                \item<4-> $v_{r+3} \in I^3[S]$.
                \item<5-> $\cdots$
                \item<5-> $v_{2r} \in H(S)$.
            \end{itemize}
        \end{column}
    \end{columns}
\end{frame}

% \begin{frame}{Para $C_n(1,r)$ tal que $3 | (r+1)$}

% \end{frame}


\begin{frame}{}
    \begin{lema}
        \label{lemma-carat-1-rx3}
        Seja $G = C_n(1,r)$ tal que $3 | (r+1)$ e $S = \{v_i \in V(G) \;|\; 3 \nmid i$ para $1\le i \le r\}$. Se $n \ge 4r - 2$ então $ v_{2r} \in  \partial H(S)$.
    \end{lema}
    
    \begin{columns}
        \begin{column}{0.7\textwidth}
            \resizebox{\textwidth}{!}{
                \begin{tikzpicture}
	\begin{pgfonlayer}{nodelayer}
		\node [style=basic, label={$v_1$}] (0) at (-4, 1.25) {};
		\node [style=basic, label={$v_2$}] (1) at (-3, 1.25) {};
		\node [style=empty, label={$v_3$}] (2) at (-2, 1.25) {};
		\node [style=basic, label={$v_4$}] (3) at (-1, 1.25) {};
		\node [style=basic, label={$v_5$}] (4) at (0, 1.25) {};
		\node [style=empty, label={$v_6$}] (16) at (1, 1.25) {};
		\node [style=basic, label={$v_7$}] (18) at (2, 1.25) {};
		\node [style=basic, label={$v_{r-1}$}] (10) at (4, 1.25) {};
		\node [style=basic, label={$v_{r}$}] (13) at (5, 1.25) {};
		%Propagação
		\node [style=empty, label={below:$v_{r+1}$}] (5) at (-4, 0) {};
		\node [style=empty, label={below:$v_{r+2}$}] (6) at (-3, 0) {};
		\node [style=empty, label={below:$v_{r+3}$}] (7) at (-2, 0) {};
		
		\node [style=empty, label={below:$v_{r+4}$}] (8) at (-1, 0) {};
		\node [style=empty, label={below:$v_{r+5}$}] (9) at (0, 0) {};
		\node [style=empty, label={below:$v_{r+6}$}] (17) at (1, 0) {};
		\node [style=empty, label={below:$v_{r+7}$}] (19) at (2, 0) {};
		%...
		\node [style=empty, label={below:$v_{2r-2}$}] (15) at (3, 0) {};
		\node [style=empty, label={below:$v_{2r-1}$}] (11) at (4, 0) {};
		\node [style=empty, label={below:$v_{2r}$}] (12) at (5, 0) {};


		\node [style=empty, label={$v_{r-2}$}] (14) at (3, 1.25) {};

		%I1
		%, label={$v_3$}, label={$v_6$}, label={$v_{r-2}$}, label={below:$v_{r+1}$}
		\node [style=cazul, visible on=<2->] () at (-2, 1.25) {};
		\node [style=cazul, visible on=<2->] () at (1, 1.25) {};
		\node [style=cazul, visible on=<2->] () at (3, 1.25) {};
		\node [style=cazul, visible on=<2->] () at (-4, 0) {};
		%I2 
		%, label={below:$v_{r+2}$}
		\node [style=cazul, visible on=<3->] () at (-3, 0) {};
		%I3
		% , label={below:$v_{r+3}$}
		\node [style=cazul, visible on=<4->] () at (-2, 0) {};
		%I
		% V2r
		\node [style=cvermelho, visible on=<5->] () at (5, 0) {};
		% Labels, label={below:$v_{r+4}$}, label={below:$v_{r+5}$}, label={below:$v_{r+6}$}
		%, label={below:$v_{r+7}$}, label={below:$v_{2r-2}$}, label={below:$v_{2r-1}$}
		\node [style=cazul, visible on=<5->] () at (-1, 0) {};
		\node [style=cazul, visible on=<5->] () at (0, 0) {};
		\node [style=cazul, visible on=<5->] () at (1, 0) {};
		\node [style=cazul, visible on=<5->] () at (2, 0) {};
		%...
		\node [style=cazul, visible on=<5->] () at (3, 0) {};
		\node [style=cazul, visible on=<5->] () at (4, 0) {};


	\end{pgfonlayer}
	\begin{pgfonlayer}{edgelayer}
		\draw (0) to (5);
		\draw (1) to (6);
		\draw (2) to (7);
		\draw (3) to (8);
		\draw (4) to (9);
		\draw (0) to (1);
		\draw (1) to (2);
		\draw (2) to (3);
		\draw (3) to (4);
		\draw (5) to (6);
		\draw (6) to (7);
		\draw (7) to (8);
		\draw (8) to (9);
		\draw (10) to (13);
		\draw (13) to (12);
		\draw (12) to (11);
		\draw (10) to (11);
		\draw (13) to (5);
		\draw (14) to (10);
		\draw (15) to (11);
		\draw (14) to (15);
		\draw (16) to (17);
		\draw (18) to (19);
		\draw (4) to (16);
		\draw (16) to (18);
		\draw (9) to (17);
		\draw (17) to (19);
		\draw[dashed] (19) to (15);
		\draw[dashed] (18) to (14);
	\end{pgfonlayer}
\end{tikzpicture}

            }
            \resizebox{\textwidth}{!}{
                \begin{tikzpicture}
	\begin{pgfonlayer}{nodelayer}
		\node [style=basic, label={$v_r$}] (0) at (-4, 1.25) {};
		\node [style=basic, label={$v_{r-1}$}] (1) at (-3, 1.25) {};
		\node [style=empty, label={$v_{r-2}$}] (2) at (-2, 1.25) {};
		\node [style=basic, label={$v_{r-3}$}] (3) at (-1, 1.25) {};
		\node [style=basic, label={$v_{r-4}$}] (4) at (0, 1.25) {};
		\node [style=empty, label={$v_3$}] (14) at (3, 1.25) {};
		\node [style=basic, label={$v_2$}] (10) at (4, 1.25) {};
		\node [style=basic, label={$v_1$}] (13) at (5, 1.25) {};
		\node [style=empty, label={below:$v_n$}] (5) at (-4, 0) {};
		\node [style=empty, label={below:$v_{n-1}$}] (6) at (-3, 0) {};
		\node [style=empty, label={below:$v_{n-2}$}] (7) at (-2, 0) {};
		\node [style=empty, label={below:$v_{n-3}$}] (8) at (-1, 0) {};
		\node [style=empty, label={below:$v_{n-4}$}] (9) at (0, 0) {};
		\node [style=empty, label={below:\rotatebox{-90}{$v_{n-(r-3)}$}}] (15) at (3, 0) {};
		\node [style=empty, label={below:\rotatebox{-90}{$v_{n-(r-2)}$}}] (11) at (4, 0) {};
		\node [style=empty, label={below:\rotatebox{-90}{$v_{n-(r-1)}$}}] (12) at (5, 0) {};
		\node [style=empty, label={$v_{r-5}$}] (16) at (1, 1.25) {};
		\node [style=empty, label={below:$v_{n-5}$}] (17) at (1, 0) {};
		\node [style=basic, label={$v_{r-6}$}] (18) at (2, 1.25) {};
		\node [style=empty, label={below:$v_{n-6}$}] (19) at (2, 0) {};
		%I1 
		%labels , label={$v_{r-2}$}, label={$v_3$}, label={$v_{r-5}$}, label={below:$v_n$}
		\node [style=camarelo, visible on=<2->] () at (-2, 1.25) {};
		\node [style=camarelo, visible on=<2->] () at (3, 1.25) {};
		\node [style=camarelo, visible on=<2->] () at (1, 1.25) {};
		\node [style=camarelo, visible on=<2->] () at (-4, 0) {};
		%I2 , label={below:$v_{n-1}$}
		\node [style=camarelo, visible on=<3->] () at (-3, 0) {};
		%I3  label={below:$v_{n-2}$}
		\node [style=camarelo, visible on=<4->]  () at (-2, 0) {};
		%I4 , label={below:\rotatebox{-90}{$v_{n-(r-1)}$}}
		\node [style=camarelo, visible on=<5->] () at (5, 0) {};
		%Labels, label={below:$v_{n-3}$}, label={below:$v_{n-4}$}, label={below:$v_{n-5}$},  label={below:\rotatebox{-90}{$v_{n-(r-3)}$}}, label={below:\rotatebox{-90}{$v_{n-(r-2)}$}}
		\node [style=camarelo, visible on=<5->] () at (-1, 0) {};
		\node [style=camarelo, visible on=<5->] () at (0, 0) {};
		\node [style=camarelo, visible on=<5->] () at (1, 0) {};
		\node [style=camarelo, visible on=<5->] () at (2, 0) {};
		\node [style=camarelo, visible on=<5->] () at (3, 0) {};
		\node [style=camarelo, visible on=<5->] () at (4, 0) {};

	\end{pgfonlayer}
	\begin{pgfonlayer}{edgelayer}
		\draw (0) to (5);
		\draw (1) to (6);
		\draw (2) to (7);
		\draw (3) to (8);
		\draw (4) to (9);
		\draw (0) to (1);
		\draw (1) to (2);
		\draw (2) to (3);
		\draw (3) to (4);
		\draw (5) to (6);
		\draw (6) to (7);
		\draw (7) to (8);
		\draw (8) to (9);
		\draw (10) to (13);
		\draw (13) to (12);
		\draw (12) to (11);
		\draw (10) to (11);
		\draw (13) to (5);
		\draw (14) to (10);
		\draw (15) to (11);
		\draw (14) to (15);
		\draw (16) to (17);
		\draw (18) to (19);
		\draw (4) to (16);
		\draw (16) to (18);
		\draw [dashed] (18) to (14);
		\draw (9) to (17);
		\draw (17) to (19);
		\draw [dashed] (19) to (15);
	\end{pgfonlayer}
\end{tikzpicture}

            }
            % Para:
            % \begin{itemize}
            %     \tightlist
            %     \item $C_n(1,r)$ com $3 | (r+1)$ e $n \ge 4\times r - 2$
            %     \item $S = \{v_i \in V(G) \;|\; 3 \nmid i$ para $1\le i \le r\}$.
            % \end{itemize}
        \end{column}
        \begin{column}{0.3\textwidth}
            % Quando $n \ge 4\times r - 2$ e $S = \{v_i \in V(G) \;|\; 3 \nmid i$ para $1\le i \le r\}$ temos:
            \footnotesize
            \begin{itemize}
                \tightlist
                \item<2-> $I^1[S]=S\cup \{ v_i \; | \; 3 \mid i \} \cup \{v_{r+1}, v_n \}$
                \item<3-> $I^2[S]=I^1[S]\cup\{v_{r+2}, v_{n-1}\}$
                \item<4-> $I^3[S]=I^2[S]\cup\{v_{r+3}, v_{n-2}\}$
                \item<5-> $I^{r}[S]=I^{r-1}[S]\cup\{v_{2r}, v_{n-(r-1)}\}$
                % \item<6-> $v_{2r} \in H(S)$ e $v_{2r} \not\in H(S-j)|\forall j\in S$
                % \item<7-> Logo $v_{2r} \in \partial H(S)$
            \end{itemize}
        \end{column}
    \end{columns}
\end{frame}


\begin{frame}{Para $G=C_n(1,r)$ tal que $3 | (r+1)$}
    \begin{teo}
        \label{theorem-carat-1-rx3}
        Seja $G = C_n(1,r)$ tal que $3 | (r+1)$ e $n \ge 4\times r - 2$ então $c(G) \ge \frac{2(r+1)}{3}$.%então $G$ possui um conjunto de Carathéodory de cardinalidade $ \frac{2(r+1)}{3} $. % \rceil \frac{2r}{3} \lceil$
    \end{teo}
    % \begin{proof}
    %     O conjunto $S = \{v_i \;|\; 3 \nmid i$ para $1\le i \le r\}$ pelo Lema~\ref{lemma-circ-1-r} temos que $v_{2r} \in H(S)$, e pelo Lema~\ref{lemma-carat-1-rx3} temos que $v_{2r} \in \partial H(S)$, portanto $S$ é um conjunto de Carathéodory de $G$. O conjunto $S$ tem cardinalidade $\frac{2(r+1)}{3}$ consequentemente o número de Carathéodory de $G$ é maior ou igual $\frac{2(r+1)}{3} $.  %$r - \lfloor \frac{r}{3} \rfloor = \rceil \frac{2r}{3} \lceil$.
    % \end{proof}
\end{frame}


\begin{frame}{Exemplo}
    \begin{figure}
        \centering
        \resizebox{\textwidth}{!}{
            \centering
            \begin{tikzpicture}
	\begin{pgfonlayer}{nodelayer}
		\node [style=empty, label={$1$}] (17) at (-7.64591, 3.27063) {};
		\node [style=basic, label={above right:$2$}] (1) at (-6.56093, 2.85267) {};
		\node [style=basic, label={right:$3$}] (0) at (-5.68223, 2.09788) {};
		\node [style=basic, label={right:$4$}] (2) at (-5.11995, 1.08449) {};
		\node [style=basic, label={right:$5$}] (16) at (-4.93823, -0.05503) {};
		\node [style=empty, label={right:$6$}] (15) at (-5.14918, -1.19562) {};
		\node [style=empty, label={below right:$7$}] (12) at (-5.73964, -2.18778) {};
		\node [style=empty, label={below right:$8$}] (11) at (-6.63631, -2.9267) {};
		\node [style=empty, label={below:$9$}] (8) at (-7.73279, -3.30467) {};
		\node [style=empty, label={below:$10$}] (6) at (-8.8849, -3.29608) {};
		\node [style=empty, label={below left:$11$}] (4) at (-9.97193, -2.8879) {};
		\node [style=empty, label={below left:$12$}] (3) at (-10.852, -2.12699) {};
		\node [style=empty, label={left:$13$}] (5) at (-11.4274, -1.11805) {};
		\node [style=empty, label={left:$14$}] (7) at (-11.6071, 0.02817) {};
		\node [style=empty, label={left:$15$}] (9) at (-11.3829, 1.16353) {};
		\node [style=empty, label={left:$16$}] (10) at (-10.7934, 2.15672) {};
		\node [style=empty, label={above left:$17$}] (13) at (-9.90118, 2.89467) {};
		\node [style=empty, label={$18$}] (14) at (-8.80853, 3.28571) {};
		\node [style=none] (18) at (-8.25, 0) {$H(S-\{1\})$};
		\node [style=basic, label={$1$}] (19) at (0.52296, 3.27063) {};
		\node [style=empty, label={above right:$2$}] (20) at (1.60794, 2.85267) {};
		\node [style=empty, label={right:$3$}] (21) at (2.48664, 2.09788) {};
		\node [style=basic, label={right:$4$}] (22) at (3.04892, 1.08449) {};
		\node [style=basic, label={right:$5$}] (23) at (3.23064, -0.05503) {};
		\node [style=basic, label={right:$6$}] (24) at (3.01969, -1.19562) {};
		\node [style=empty, label={below right:$7$}] (25) at (2.42923, -2.18778) {};
		\node [style=empty, label={below right:$8$}] (26) at (1.53256, -2.9267) {};
		\node [style=empty, label={below:$9$}] (27) at (0.43608, -3.30467) {};
		\node [style=empty, label={below:$10$}] (28) at (-0.71603, -3.29608) {};
		\node [style=empty, label={below left:$11$}] (29) at (-1.80303, -2.8879) {};
		\node [style=empty, label={below left:$12$}] (30) at (-2.68313, -2.12699) {};
		\node [style=empty, label={left:$13$}] (31) at (-3.25853, -1.11805) {};
		\node [style=empty, label={left:$14$}] (32) at (-3.43823, 0.02817) {};
		\node [style=empty, label={left:$15$}] (33) at (-3.21403, 1.16353) {};
		\node [style=empty, label={left:$16$}] (34) at (-2.62453, 2.15672) {};
		\node [style=basic, label={above left:$17$}] (35) at (-1.73233, 2.89467) {};
		\node [style=basic, label={$18$}] (36) at (-0.63966, 3.28571) {};
		\node [style=none] (37) at (-0.08113, 0) {$H(S-\{2\})$};
		\node [style=basic, label={$1$}] (38) at (8.69183, 3.27063) {};
		\node [style=basic, label={above right:$2$}] (39) at (9.77681, 2.85267) {};
		\node [style=empty, label={right:$3$}] (40) at (10.6555, 2.09788) {};
		\node [style=empty, label={right:$4$}] (41) at (11.2178, 1.08449) {};
		\node [style=basic, label={right:$5$}] (42) at (11.3995, -0.05503) {};
		\node [style=basic, label={right:$6$}] (43) at (11.1886, -1.19562) {};
		\node [style=basic, label={below right:$7$}] (44) at (10.5981, -2.18778) {};
		\node [style=empty, label={below right:$8$}] (45) at (9.70143, -2.9267) {};
		\node [style=empty, label={below:$9$}] (46) at (8.60495, -3.30467) {};
		\node [style=empty, label={below:$10$}] (47) at (7.45284, -3.29608) {};
		\node [style=empty, label={below left:$11$}] (48) at (6.36584, -2.8879) {};
		\node [style=empty, label={below left:$12$}] (49) at (5.48574, -2.12699) {};
		\node [style=empty, label={left:$13$}] (50) at (4.91034, -1.11805) {};
		\node [style=empty, label={left:$14$}] (51) at (4.73064, 0.02817) {};
		\node [style=empty, label={left:$15$}] (52) at (4.95484, 1.16353) {};
		\node [style=empty, label={left:$16$}] (53) at (5.54434, 2.15672) {};
		\node [style=empty, label={above left:$17$}] (54) at (6.43654, 2.89467) {};
		\node [style=empty, label={$18$}] (55) at (7.52921, 3.28571) {};
		\node [style=none] (56) at (8.08774, 0) {$H(S-\{4\})$};
	\end{pgfonlayer}
	\begin{pgfonlayer}{edgelayer}
		\draw (17) to (1);
		\draw (1) to (0);
		\draw (0) to (2);
		\draw (2) to (16);
		\draw (16) to (15);
		\draw (15) to (12);
		\draw (12) to (11);
		\draw (11) to (8);
		\draw (8) to (6);
		\draw (6) to (4);
		\draw (4) to (3);
		\draw (3) to (5);
		\draw (5) to (7);
		\draw (7) to (9);
		\draw (9) to (10);
		\draw (10) to (13);
		\draw (13) to (14);
		\draw (14) to (17);
		\draw (17) to (15);
		\draw (1) to (12);
		\draw (0) to (11);
		\draw (2) to (8);
		\draw (16) to (6);
		\draw (15) to (4);
		\draw (12) to (3);
		\draw (11) to (5);
		\draw (8) to (7);
		\draw (6) to (9);
		\draw (4) to (10);
		\draw (3) to (13);
		\draw (5) to (14);
		\draw (7) to (17);
		\draw (9) to (1);
		\draw (10) to (0);
		\draw (13) to (2);
		\draw (19) to (20);
		\draw (20) to (21);
		\draw (21) to (22);
		\draw (22) to (23);
		\draw (23) to (24);
		\draw (24) to (25);
		\draw (25) to (26);
		\draw (26) to (27);
		\draw (27) to (28);
		\draw (28) to (29);
		\draw (29) to (30);
		\draw (30) to (31);
		\draw (31) to (32);
		\draw (32) to (33);
		\draw (33) to (34);
		\draw (34) to (35);
		\draw (35) to (36);
		\draw (36) to (19);
		\draw (19) to (24);
		\draw (20) to (25);
		\draw (21) to (26);
		\draw (22) to (27);
		\draw (23) to (28);
		\draw (24) to (29);
		\draw (25) to (30);
		\draw (26) to (31);
		\draw (27) to (32);
		\draw (28) to (33);
		\draw (29) to (34);
		\draw (30) to (35);
		\draw (31) to (36);
		\draw (32) to (19);
		\draw (33) to (20);
		\draw (34) to (21);
		\draw (35) to (22);
		\draw (38) to (39);
		\draw (39) to (40);
		\draw (40) to (41);
		\draw (41) to (42);
		\draw (42) to (43);
		\draw (43) to (44);
		\draw (44) to (45);
		\draw (45) to (46);
		\draw (46) to (47);
		\draw (47) to (48);
		\draw (48) to (49);
		\draw (49) to (50);
		\draw (50) to (51);
		\draw (51) to (52);
		\draw (52) to (53);
		\draw (53) to (54);
		\draw (54) to (55);
		\draw (55) to (38);
		\draw (38) to (43);
		\draw (39) to (44);
		\draw (40) to (45);
		\draw (41) to (46);
		\draw (42) to (47);
		\draw (43) to (48);
		\draw (44) to (49);
		\draw (45) to (50);
		\draw (46) to (51);
		\draw (47) to (52);
		\draw (48) to (53);
		\draw (49) to (54);
		\draw (50) to (55);
		\draw (51) to (38);
		\draw (52) to (39);
		\draw (53) to (40);
		\draw (54) to (41);
		\draw (14) to (16);
		\draw (36) to (23);
		\draw (55) to (42);
	\end{pgfonlayer}
\end{tikzpicture}

        }
        \caption{Exemplo de um conjunto de Carathéodory para $C_{18}(1,5)$}
        \label{fig-c18-pt1}
    \end{figure}
\end{frame}


\begin{frame}{Exemplo}
    \begin{figure}
        \centering
        \resizebox{\textwidth}{!}{
            \centering
            \begin{tikzpicture}%[scale=0.6]
	\begin{pgfonlayer}{nodelayer}
		\node [style=basic, label={$1$}] (17) at (-7.64591, 3.27063) {};
		\node [style=basic, label={above right:$2$}] (1) at (-6.56093, 2.85267) {};
		\node [style=basic, label={right:$3$}] (0) at (-5.68223, 2.09788) {};
		\node [style=basic, label={right:$4$}] (2) at (-5.11995, 1.08449) {};
		\node [style=empty, label={right:$5$}] (16) at (-4.93823, -0.05503) {};
		\node [style=empty, label={right:$6$}] (15) at (-5.14918, -1.19562) {};
		\node [style=empty, label={below right:$7$}] (12) at (-5.73964, -2.18778) {};
		\node [style=empty, label={below right:$8$}] (11) at (-6.63631, -2.9267) {};
		\node [style=empty, label={below:$9$}] (8) at (-7.73279, -3.30467) {};
		\node [style=empty, label={below:$10$}] (6) at (-8.8849, -3.29608) {};
		\node [style=empty, label={below left:$11$}] (4) at (-9.97193, -2.8879) {};
		\node [style=empty, label={below left:$12$}] (3) at (-10.852, -2.12699) {};
		\node [style=empty, label={left:$13$}] (5) at (-11.4274, -1.11805) {};
		\node [style=empty, label={left:$14$}] (7) at (-11.6071, 0.02817) {};
		\node [style=empty, label={left:$15$}] (9) at (-11.3829, 1.16353) {};
		\node [style=empty, label={left:$16$}] (10) at (-10.7934, 2.15672) {};
		\node [style=empty, label={above left:$17$}] (13) at (-9.90118, 2.89467) {};
		\node [style=empty, label={$18$}] (14) at (-8.80853, 3.28571) {};
		\node [style=none] (18) at (-8.25, 0) {$H(S-\{5\})$};
		\node [style=basic, label={$1$}] (19) at (0.52296, 3.27063) {};
		\node [style=basic, label={above right:$2$}] (20) at (1.60794, 2.85267) {};
		\node [style=basic, label={right:$3$}] (21) at (2.48664, 2.09788) {};
		\node [style=basic, label={right:$4$}] (22) at (3.04892, 1.08449) {};
		\node [style=basic, label={right:$5$}] (23) at (3.23064, -0.05503) {};
		\node [style=basic, label={right:$6$}] (24) at (3.01969, -1.19562) {};
		\node [style=basic, label={below right:$7$}] (25) at (2.42923, -2.18778) {};
		\node [style=basic, label={below right:$8$}] (26) at (1.53256, -2.9267) {};
		\node [style=basic, label={below:$9$}] (27) at (0.43608, -3.30467) {};
		\node [style=basic, label={below:$10$}] (28) at (-0.71603, -3.29608) {};
		\node [style=basic, label={below left:$11$}] (29) at (-1.80303, -2.8879) {};
		\node [style=basic, label={below left:$12$}] (30) at (-2.68313, -2.12699) {};
		\node [style=basic, label={left:$13$}] (31) at (-3.25853, -1.11805) {};
		\node [style=basic, label={left:$14$}] (32) at (-3.43823, 0.02817) {};
		\node [style=basic, label={left:$15$}] (33) at (-3.21403, 1.16353) {};
		\node [style=basic, label={left:$16$}] (34) at (-2.62453, 2.15672) {};
		\node [style=basic, label={above left:$17$}] (35) at (-1.73233, 2.89467) {};
		\node [style=basic, label={$18$}] (36) at (-0.63966, 3.28571) {};
		\node [style=none] (37) at (-0.08113, 0) {$H(S)$};
		\node [style=empty, label={$1$}] (38) at (8.69183, 3.27063) {};
		\node [style=empty, label={above right:$2$}] (39) at (9.77681, 2.85267) {};
		\node [style=empty, label={right:$3$}] (40) at (10.6555, 2.09788) {};
		\node [style=empty, label={right:$4$}] (41) at (11.2178, 1.08449) {};
		\node [style=empty, label={right:$5$}] (42) at (11.3995, -0.05503) {};
		\node [style=empty, label={right:$6$}] (43) at (11.1886, -1.19562) {};
		\node [style=empty, label={below right:$7$}] (44) at (10.5981, -2.18778) {};
		\node [style=basic, label={below right:$8$}] (45) at (9.70143, -2.9267) {};
		\node [style=basic, label={below:$9$}] (46) at (8.60495, -3.30467) {};
		\node [style=basic, label={below:$10$}] (47) at (7.45284, -3.29608) {};
		\node [style=basic, label={below left:$11$}] (48) at (6.36584, -2.8879) {};
		\node [style=basic, label={below left:$12$}] (49) at (5.48574, -2.12699) {};
		\node [style=basic, label={left:$13$}] (50) at (4.91034, -1.11805) {};
		\node [style=basic, label={left:$14$}] (51) at (4.73064, 0.02817) {};
		\node [style=basic, label={left:$15$}] (52) at (4.95484, 1.16353) {};
		\node [style=basic, label={left:$16$}] (53) at (5.54434, 2.15672) {};
		\node [style=empty, label={above left:$17$}] (54) at (6.43654, 2.89467) {};
		\node [style=empty, label={$18$}] (55) at (7.52921, 3.28571) {};
		\node [style=none] (56) at (8.08774, 0) {$\partial H(S)$};
	\end{pgfonlayer}
	\begin{pgfonlayer}{edgelayer}
		\draw (17) to (1);
		\draw (1) to (0);
		\draw (0) to (2);
		\draw (2) to (16);
		\draw (16) to (15);
		\draw (15) to (12);
		\draw (12) to (11);
		\draw (11) to (8);
		\draw (8) to (6);
		\draw (6) to (4);
		\draw (4) to (3);
		\draw (3) to (5);
		\draw (5) to (7);
		\draw (7) to (9);
		\draw (9) to (10);
		\draw (10) to (13);
		\draw (13) to (14);
		\draw (14) to (17);
		\draw (17) to (15);
		\draw (1) to (12);
		\draw (0) to (11);
		\draw (2) to (8);
		\draw (16) to (6);
		\draw (15) to (4);
		\draw (12) to (3);
		\draw (11) to (5);
		\draw (8) to (7);
		\draw (6) to (9);
		\draw (4) to (10);
		\draw (3) to (13);
		\draw (5) to (14);
		\draw (7) to (17);
		\draw (9) to (1);
		\draw (10) to (0);
		\draw (13) to (2);
		\draw (19) to (20);
		\draw (20) to (21);
		\draw (21) to (22);
		\draw (22) to (23);
		\draw (23) to (24);
		\draw (24) to (25);
		\draw (25) to (26);
		\draw (26) to (27);
		\draw (27) to (28);
		\draw (28) to (29);
		\draw (29) to (30);
		\draw (30) to (31);
		\draw (31) to (32);
		\draw (32) to (33);
		\draw (33) to (34);
		\draw (34) to (35);
		\draw (35) to (36);
		\draw (36) to (19);
		\draw (19) to (24);
		\draw (20) to (25);
		\draw (21) to (26);
		\draw (22) to (27);
		\draw (23) to (28);
		\draw (24) to (29);
		\draw (25) to (30);
		\draw (26) to (31);
		\draw (27) to (32);
		\draw (28) to (33);
		\draw (29) to (34);
		\draw (30) to (35);
		\draw (31) to (36);
		\draw (32) to (19);
		\draw (33) to (20);
		\draw (34) to (21);
		\draw (35) to (22);
		\draw (38) to (39);
		\draw (39) to (40);
		\draw (40) to (41);
		\draw (41) to (42);
		\draw (42) to (43);
		\draw (43) to (44);
		\draw (44) to (45);
		\draw (45) to (46);
		\draw (46) to (47);
		\draw (47) to (48);
		\draw (48) to (49);
		\draw (49) to (50);
		\draw (50) to (51);
		\draw (51) to (52);
		\draw (52) to (53);
		\draw (53) to (54);
		\draw (54) to (55);
		\draw (55) to (38);
		\draw (38) to (43);
		\draw (39) to (44);
		\draw (40) to (45);
		\draw (41) to (46);
		\draw (42) to (47);
		\draw (43) to (48);
		\draw (44) to (49);
		\draw (45) to (50);
		\draw (46) to (51);
		\draw (47) to (52);
		\draw (48) to (53);
		\draw (49) to (54);
		\draw (50) to (55);
		\draw (51) to (38);
		\draw (52) to (39);
		\draw (53) to (40);
		\draw (54) to (41);
		\draw (14) to (16);
		\draw (36) to (23);
		\draw (55) to (42);
	\end{pgfonlayer}
\end{tikzpicture}

        }
        \caption{Exemplo de um conjunto de Carathéodory para $C_{18}(1,5)$}
        \label{fig-c18-pt2}
    \end{figure}
\end{frame}


\begin{frame}{Para $C_n(1,r)$ tal que $3\nmid(r+1)$}
    De forma análoga vamos concluir um limite inferior para o número de Carathéodory para $C_n(1,r)$, onde $r+1$ não é múltiplo de 3.
\end{frame}

% \begin{frame}{Para $C_n(1,r)$ tal que $3\nmid(r+1)$}
%     \begin{figure}
%     % \resizebox{\textwidth}{!}{
%         \centering
%         \begin{tikzpicture}
	\begin{pgfonlayer}{nodelayer}
		\node [style=basic, label={$v_1$}] (0) at (-3, 5) {};
		\node [style=basic, label={below:$v_2$}] (1) at (-3, 3.5) {};
		\node [style=empty, label={below:$v_3$}] (2) at (-2, 3.5) {};
		\node [style=basic, label={below:$v_4$}] (3) at (-1, 3.5) {};
		\node [style=empty, label={$v_5$}] (4) at (-1, 5) {};
		\node [style=basic, label={$v_1$}] (5) at (-3, 2) {};
		\node [style=basic, label={below:$v_2$}] (6) at (-3, 0.5) {};
		\node [style=empty, label={below:$v_3$}] (7) at (-2, 0.5) {};
		\node [style=basic, label={below:$v_4$}] (8) at (-1, 0.5) {};
		\node [style=empty, label={below:$v_5$}] (9) at (0, 0.5) {};
		\node [style=basic, label={$v_6$}] (10) at (0, 2) {};
		\node [style=empty, label={$v_7$}] (11) at (-1.5, 2) {};
		\node [style=empty, label={$v_1$}] (12) at (1, 2) {};
		\node [style=empty, label={below:$v_2$}] (13) at (1, 0.5) {};
		\node [style=empty, label={below:$v_3$}] (14) at (2, 0.5) {};
		\node [style=empty, label={below:$v_4$}] (15) at (3, 0.5) {};
		\node [style=empty, label={below:$v_5$}] (16) at (4, 0.5) {};
		\node [style=empty, label={below:$v_6$}] (17) at (4, 2) {};
		\node [style=empty, label={$v_7$}] (18) at (3, 2) {};
		\node [style=empty, label={$v_8$}] (19) at (2, 2) {};
		\node [style=empty, label={$v_1$}] (21) at (0, 5) {};
		\node [style=empty, label={below:$v_2$}] (22) at (0, 3.5) {};
		\node [style=empty, label={below:$v_3$}] (23) at (1, 3.5) {};
		\node [style=empty, label={below:$v_4$}] (24) at (2, 3.5) {};
		\node [style=empty, label={below:$v_5$}] (25) at (3, 3.5) {};
		\node [style=empty, label={below:$v_6$}] (26) at (4, 3.5) {};
		\node [style=empty, label={below:$v_7$}] (27) at (4, 5) {};
		\node [style=empty, label={$v_8$}] (28) at (3, 5) {};
		\node [style=empty, label={$v_9$}] (29) at (2, 5) {};
		\node [style=empty, label={$v_10$}] (30) at (1, 5) {};
		\node [style=none, label={$r4$}] (32) at (-2.5, 4) {};
		\node [style=none, label={$r6$}] (33) at (-2.5, 1) {};
		\node [style=none, label={$r7$}] (34) at (1.5, 1) {};
		\node [style=none, label={$r9$}] (35) at (0.5, 4) {};
	\end{pgfonlayer}
	\begin{pgfonlayer}{edgelayer}
		\draw (0) to (1);
		\draw (1) to (2);
		\draw (2) to (3);
		\draw (3) to (4);
		\draw (4) to (0);
		\draw (5) to (6);
		\draw (6) to (7);
		\draw (7) to (8);
		\draw (8) to (9);
		\draw (9) to (10);
		\draw (10) to (11);
		\draw (11) to (5);
		\draw (12) to (13);
		\draw (13) to (14);
		\draw (14) to (15);
		\draw (15) to (16);
		\draw (16) to (17);
		\draw (17) to (18);
		\draw (18) to (19);
		\draw (21) to (22);
		\draw (22) to (23);
		\draw (23) to (24);
		\draw (24) to (25);
		\draw (25) to (26);
		\draw (26) to (27);
		\draw (27) to (28);
		\draw (28) to (29);
		\draw (29) to (30);
		\draw (12) to (19);
		\draw (30) to (21);
	\end{pgfonlayer}
\end{tikzpicture}

%     % }
%     \caption{Exemplos de $S$ para quando $3\nmid (r+1)$ 
%     %conforme Lema\ref{lemma-carat-1-r-geral}
%     }
%     \label{fig-exemplo-lema-geral}
% \end{figure}
% \end{frame}

\begin{frame}{Para $C_n(1,r)$ tal que $3\nmid(r+1)$}
\begin{lema}
        \label{lemma-carat-1-r-geral}
        Seja $G=C_n(1,r)$ tal que $3\nmid(r+1)$ e $S\subseteq V(G)$. Se $S=\{v_r\}\cup \{ v_i | 3 \nmid i $ tal que $1\le i  < r-1 \}$ então $v_{2r} \in H(S)$.
    \end{lema}
    \begin{lema}
        \label{lemma-carat-1-rx3-geral}
        Seja $G = C_n(1,r)$ tal que $3 \nmid (r+1)$ e $S=\{ v_i | 3 \nmid i $ tal que $1\le i  < r-1 \}\cup \{v_r\}$  . Se $n \ge 4r - 2$ então $ v_{2r} \in  \partial H(S)$.
    \end{lema}

    \begin{teo}
        \label{theorem-carat-1-r-geral}
        Seja $G = C_n(1,r)$ tal que $3 \nmid (r+1)$ e $n \ge 4\times r - 2$  então $c(G) \ge \lfloor \frac{2r+1}{3} \rfloor$. %$G$ possui um conjunto de Carathéodory de cardinalidade $\lfloor \frac{2r+1}{3} \rfloor$. %$\frac{2r+1}{3}$.
    \end{teo}
\end{frame}